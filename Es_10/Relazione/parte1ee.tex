\section{Scopo e strumenti}
Lo scopo di questa esperienza è di osservare il funionamento della porta logica not per diversi segnali in ingresso e studiare anche il suo comportamento in regime non ideale.\\
Gli strumenti a disposizione includono:
\begin{itemize}
\item{Integrato IC SN74LS04}
\item{IC SN74LS244 octal buffer/driver}
\item{arduino nano}
\item{trimmer da 2K$\Omega$, 10K$\Omega$, 100K$\Omega$}
\end{itemize}
Nella prima parte è stato analizzata la porta logica in condizioni statiche, nella seconda invece in condizioni dinamiche.\\

\section{Caratteristiche statiche}
E' stato montato il circuito in \fig{circuito1}, composto dalla resistenza $R_{2}=100 7pm 1 /Omega$; trimmer da 1.98$\pm 0.02 k\Omega$ ed una delle porte not contenute nell'integrato SN74LS04.\\
 Agendo sul trimmer "R1" è stato possibile far variare il potenziale in ingresso alla porta Not e ne è stata osservata l'uscita. Il potenziale di alimentazione era $V_{cc}$=4.29 $\pm$ 0.03 V, mentre il potenziale in ingresso alla porta  variava tra $V_{IN,min}=0.0 \pm 0.1$ mV e $V_{IN,max}=4.30 \pm 0.03$ V. Al variare del potenziale in ingresso è stato realizzato un grafico di $V_{OUT}$ in funzione di $V_{IN}$: \fig{out_in}. Dal grafico si può dare una stima dei potenziali in ingresso che corrispondono al massimo valore per cui il segnale è interpretato logicamente come "0" ($V_{IL}$) e il minimo per cui è interpretato come "1" ($V_{IH}$). Si è considerato come $V_{IL}$ il potenziale massimo per cui l'uscita si attesta sul suo valore massimo, prima di iniziare a calare in funzione dell'ingresso. Invece come $V_{IH}$ il valore di potenziale per cui la decrescita dell'uscita si arresta, ovvero il minimo valore per cui l'uscita è a potenziale basso. I valori stimati sono, come è possibile dedurre dal frafico \fig{out_in}  $V_{IL}=??$ e $V_{IH}$=??. 
 E' stato possibile stimare queste due grandezze anche dall'osservazione delle tensioni tramite l'oscilloscopio, variando gradualmente il rapporto di partizione del trimmer. Nelle zone di transizione il potenziale in ingresso  non viene riconosciuto come segnale logico e il segnale in uscita si stabilizza su un valore intermedio compreso tra i valori alto e basso. I valori di  $V_{IL}$ e $V_{IH}$ stimati sono in accordo con le informazioni riportate sul datasheet (\ref(t:statica))  \\
\begin{table}[h]
	\centering
	\begin{tabular}{*{4}{S[table-figures-exponent = 2]} }
		{} & {Valore misurato} & {Incertezza} & {absolute maximum ratings} \\
		\midrule
    $V_{IH}$ & 1.5 V & 0.1 V\\
    $V_{IL}$ & 1.0 V & 0.1 V\\
    $V_{OH}$ & 4.1 V & 0.1 V\\
    $V_{OL}$ & 114 mV & 4 mV\\
    $I_{IH}$ & 158 $\mu$A &  2$\mu$\\
    $I_{IL}$ & 195 $\mu$A& 3 $\mu$A
    $I_{OH}$ & 37 $\mu$A& 1 $\mu$A&
    $I_{OL}$ & 
	
  
 	\end{tabular}
	\caption{ Le misure di potenziale sono state effettuate tramite oscilloscopio. Quelle di corrente di ingresso tramite amperometro digitale (multimetro) posto tra il trimmer e l'ingresso della porta. Le correnti di uscita invece indirettamente misurando la tensione ai capi della resistenza $R_{2}$. Gli errori per i potenziali in ingresso sono sensibilmente maggiori delle risoluzione degli strumenti utilizzati, in quanto non era definita nettamente la transizione tra due regimi di lavoro. In uscita invece sono dovuti alla risoluzione degli strumenti come somma in quadratura di errore di calibrazione e e incertezza di lettura}
	\label{t:tabella2}
\end{table}