\begin{section}*{Scopo e strumentazione}

L'esperienza, divisa in due parti, è volta dapprima alla misura della lunghezza d'onda della riga gialla del sodio, a partire dall'angolo di diffrazione, quindi viene stimata la costante di Rydberg a partire dalle lunghezze d'onda delle righe dell'idrogeno.\\ La strumentazione include uno spettroscopio a prisma (utilizzato nella prima parte), la cui struttura è mostrata in \emph{Figura 1}, uno spettroscopio a reticolo di diffrazione (\emph{Figura 2}, utilizzato nella seconda), una lampada al cadmio, una lampada al mercurio, una lampada all'idrogeno ed una lampada al sodio.

%\begin{figure}[h]
%\begin{center}
%\includegraphics[width=5cm]{}
%\caption{spettroscopio a prisma}
%\end{center}
%\end{figure}

\end{section}

\begin{section}*{Misura della lunghezza d'onda della riga gialla del sodio}

\begin{subsection}*{Calibrazione dello strumento con la lampada al cadmio}

Abbiamo posto la lampada al cadmio nella posizione indicata in \emph{Figura 1}, in modo tale che la luce investisse completamente la fenditura. Si è tolto quindi il prisma, facendo attenzione a non toccarlo (avvalendoci dell'uso della vite) e abbiamo allineato i due telescopi, ricercando la fenditura e centrandola. Leggendo il goniometro abbiamo così potuto determinare un angolo di riferimento. La misura è stata svolta separatamente dai diversi componenti del gruppo, come sempre anche in seguito per le misure di angoli, ottenendo così tre valori da cui sono state ricavate media e scarto quadratico medio, pari a $\theta_{0} = 12 \pm 2$. Si hanno infatti tre fonti di errore: l'errore casuale così calcolato e gli errori strumentali dovuti alla risoluzione del goniometro (pari a un primo) e alla larghezza della fenditura.\\ %come si trattano?
Ricollocato il prisma nella sua posizione originaria, abbiamo dunque ruotato lo stesso in modo da formare un angolo di almeno 60 gradi con la normale. Ruotando il telescopio, abbiamo allora ricercato le righe spettrali e, agendo ancora sul prisma, abbiamo raggiunto la posizione di minima deviazione per la riga verde. La posizione del prisma non è più stata variata in seguito. Abbiamo così misurato gli angoli per le righe di emissione blu, azzurra, verde e rossa (\emph{Tabella 1}), realizzando un grafico della differenza tra gli stessi e $\theta_{0}$ in funzione dell'inverso della lunghezza d'onda, nominale, mostrato in \emph{Figura 3}. Si è svolto così un fit lineare a due parametri, che dà come risultati  \\ %fit

\end{subsection}

\begin{subsection}*{Misura con la lampada al sodio}

Sostituita la lampada al cadmio con la lampada al sodio, abbiamo misurato l'angolo a cui si è osservata la banda gialla, pari a $\theta_{s} = 23 \pm 4$ rad. Utilizzando i parametri determinati nella sezione precedente, abbiamo così potuto calcolare il valore della lunghezza d'onda della riga gialla del sodio $\lambda = 598 \pm 11$ nm.

\end{subsection}

\end{section}

\begin{section}*{Misura della costante di Rydberg}

\begin{subsection}*{Misura del passo reticolare}

Prima di misurare la costante di Rydberg, abbiamo misurato il passo reticolare $d$ per lo spettroscopio a reticolo di diffrazione utilizzando la lampada a mercurio, noto $$d = \frac{m\lambda}{sen\theta_{i} - sen\theta_{d}}$$ dove $m$ è l'ordine di diffrazione, $\theta_{i}$ l'angolo di incidenza e $\theta_{d}$ l'angolo di diffrazione. Le relazioni tra questi angoli e la differenza tra gli angoli concretamente misurati e il rideterminato $\theta_{0} = 44 \pm 9$ (stavolta la risoluzione del goniometro è di mezzo primo) $\alpha_{0}$ e $\alpha_{1}$ sono $\theta_{i} = \frac{1}{2} (\pi - \alpha_{0})$ e $\theta_{d} = \pi - \theta_{i} - \alpha_{i}$.\\  Gli angoli misurati sono quelli corrispondenti alla riflessione ($\alpha_{0}$) e al primo ordine di rifrazione per la riga verde ($\alpha_{i}$) per i colori. Noti gli angoli e la lunghezza d'onda, abbiamo così determinato $d = 3$.

\end{subsection}

\begin{subsection}*{Righe spettrali dell'idrogeno}

Sostituendo alla lampada a mercurio la lampada a idrogeno, abbiamo misurato l'angolo di rifllessione e gli angoli di rifrazione per le bande viola e azzurra (primi due ordini) e per le bande verde e rossa (solo primo ordine). Qualora vi fosse più di una banda dello stesso colore, abbiamo misurato l'angolo per la più evidente. Noto $d$, abbiamo quindi calcolato $\lambda$. I risultati sono riportati in \emph{Tabella 2}.\\

\begin{table}[h]
\begin{center}
\begin{tabular}{| r @{$\pm$} l  @{ Rad } | r @{$\pm$} l  @{\quad nm } |}

\hline
\multicolumn{2}{c|}{$\alpha_{i}$} & \multicolumn{2}{c|}{$\lambda$} \\
\hline

\hline

\end{tabular}
\caption{spettro lampada a idrogeno.}
\end{center}
\end{table}

Nota l'equazione di Rydberg $$ \frac{1}{\lambda} = R (\frac{1}{n_{1}^2} - \frac{1}{n_{2}^2}) $$ dove $n_1$ e $n_2$ sono stati stimati a partire da $\lambda$ misurata e dal valore noto di R, abbiamo svolto un fit ottenendo ...\\

\end{subsection}

% sodio

\end{section}


\end{document}


