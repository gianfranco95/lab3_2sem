\section{Generatore di sequenza pseudo casuali}
E' stato utilizzato il circuito della sezione precedente ed è stata apportata una modifica, inserendo uno XOR appartenente all'integrato SN74LS86: \fig{gen_pseudocasuali}. All'ingresso dello XOR sono state inviate le uscite del terzo e del quarto D-latch, e la sua uscita è stata inviata all'ingresso DATA del primo flip flop.\\
E' stata dunque osservata la sequenza dei valori in uscita nel tempo, tramite l'osservazione dei diodi led inviando un clock a bassa frequenza $\sim$ 1 Hz. La successione osservata è riportata in \tab{sequenza}. I segnali Q1,Q2,Q3,Q4 sono prelevati rispettivamente agli ingressi dei diodi D1,D2,D3,D4.

\begin{table}[h]
	\centering
	\begin{tabular}{cccc}
		Q1 & Q2 & Q3 & Q4 \\
		\midrule
		 1 & 1 & 1 & 1\\
		 0 & 1 & 1 & 1\\
		 0 & 0 & 1 & 1\\
		 0 & 0 & 0 & 1\\
		 1 & 0 & 0 & 0\\
	 	 0 & 1 & 0 & 0\\
	 	 0 & 0 & 1 & 0\\
	 	 1 & 0 & 0 & 1\\
	 	 1 & 1 & 0 & 0\\
	 	 0 & 1 & 1 & 0\\
	 	 1 & 0 & 1 & 1\\
	 	 0 & 1 & 0 & 1\\
	 	 1 & 0 & 1 & 0\\
	 	 1 & 1 & 0 & 1\\
	 	 1 & 1 & 1 & 0\\
		 
 	\end{tabular}
	\caption{ Sequenza pseudocasuale osservata }
	\label{t:sequenza}
\end{table} 
La successione ha periodo 15 ed è completa.\\
Sono state generate altre sequenze pseudo-casuali. In particolare sono stati prelevati gli ingressi dello XOR non dalle uscite 3 e 4 ma da quelle 1 e 4. Anche in questo caso si osserva una sequenza completa. Si osservano invece sequenze non complete (periodo minore di 15) prelevando gli ingressi dalle uscite 2 e 4, oppure da 2 e 3 o da 1 e 2.\\
Infine è stato realizzato un tap usando una porta NAND i cui ingressi sono stati presi da Q3 e Q4. Anche in questo caso si osserva una sequenza completa, anche se non può essere considerata pseudocasuale in quanto ad ogni colpo di clock cambia stato una sola uscita e ciò avviene ordinatamente dalla porta Q1 a quella Q4.

\section{Conclusioni}
E' stato osservato il comportamento dei flip flop. Nella prima parte è stato analizzato il comportamento interno del flip flop d-latch. In seguito sono stati utilizzati flip flop d-latch e jk per realizzare circuiti più complessi. In tutti i casi è stato osservato un comportamento che risulta in accordo con quanto atteso.  