\section{Scopo e strumentazione}
Scopo dell'esperienza � la misura della costante di assorbimento del mylar ,utilizzando una serie di lastre dello stesso materiale su cui si � fatta incidere una luce rossa prodotta da un led a diodo. A tal fine � stato necessario l'uso di un amplificatore sincrono sensibile alla fase.
Il circuito � stato montato a blocchi e l'ingresso del circuito completo � stato prelevato dal generatore d'onda ,corrispondente ad un onda sinusoidale di ampiezza picco-picco $5.80 \pm 0.04 [V]$ e frequenza $1.04 \pm 0.01 [kHz]$.
Oltre a varie resistenze e condensatori il cui valore sar� indicato nel seguito, sono stati usati 4 TL081:JFET input op-amp, un TL082: JFET input dualop-amp, un SN7400:quad NAND gates, un DG441:quad CMOS analog switch, poi due NPN transistor modelli 2N1711, BC182, un led rosso ed un fotodiodo.
Abbiamo inoltre collegato tra loro le linee di massa onde minimizzare differenze di tensione tra le stesse.
