\begin{section}*{Misura della costante di Rydberg}

\begin{subsection}*{Misura del passo reticolare}

Prima di misurare la costante di Rydberg, abbiamo misurato il passo reticolare $d$ per lo spettroscopio a reticolo di diffrazione utilizzando la lampada a mercurio, noto $$d = \frac{m\lambda}{sen\theta_{i} - sen\theta_{d}}$$ dove $m$ è l'ordine di diffrazione, $\theta_{i}$ l'angolo di incidenza e $\theta_{d}$ l'angolo di diffrazione. Le relazioni tra questi angoli e la differenza tra gli angoli concretamente misurati e il rideterminato $\theta_{0} = 44 \pm 9$ (stavolta la risoluzione del goniometro è di mezzo primo) $\alpha_{0}$ e $\alpha_{1}$ sono $\theta_{i} = \frac{1}{2} (\pi - \alpha_{0})$ e $\theta_{d} = \pi - \theta_{i} - \alpha_{i}$.\\  Gli angoli misurati sono quelli corrispondenti alla riflessione ($\alpha_{0}$) e al primo ordine di rifrazione per la riga verde ($\alpha_{i}$) per i colori. Noti gli angoli e la lunghezza d'onda, abbiamo così determinato $d = 3$.

\end{subsection}

\begin{subsection}*{Righe spettrali dell'idrogeno}

Sostituendo alla lampada a mercurio la lampada a idrogeno, abbiamo misurato l'angolo di rifllessione e gli angoli di rifrazione per le bande viola e azzurra (primi due ordini) e per le bande verde e rossa (solo primo ordine). Qualora vi fosse più di una banda dello stesso colore, abbiamo misurato l'angolo per la più evidente. Noto $d$, abbiamo quindi calcolato $\lambda$. I risultati sono riportati in \emph{Tabella 2}.\\

\begin{table}[h]
\begin{center}
\begin{tabular}{| r @{$\pm$} l  @{ Rad } | r @{$\pm$} l  @{\quad nm } |}

\hline
\multicolumn{2}{c|}{$\alpha_{i}$} & \multicolumn{2}{c|}{$\lambda$} \\
\hline

\hline

\end{tabular}
\caption{spettro lampada a idrogeno.}
\end{center}
\end{table}

Nota l'equazione di Rydberg $$ \frac{1}{\lambda} = R (\frac{1}{n_{1}^2} - \frac{1}{n_{2}^2}) $$ dove $n_1$ e $n_2$ sono stati stimati a partire da $\lambda$ misurata e dal valore noto di R, abbiamo svolto un fit ottenendo ...\\

\end{subsection}

% sodio

\end{section}