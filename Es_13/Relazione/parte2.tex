\section{Macchina a stati finiti con arduino}
E' stato realizzato lo stesso semaforo tramite arduino.
Sono state stabilite come porte di output le porte D9, D10, D11 che sono state poste all'ingresso dei diodi rispettivamente verde, giallo e rosso. Le uscite sono state prelevate non direttamente da arduino ma dal buffer 74LS244 ad esso connesso. Inoltre in serie ai diodi sono state lasciate le resistenze di limitazione della corrente, utilizzate anche in precedenza.\\
E' stato stabilita come input-pullup la porta D8, che può assumere in ingresso i due valori di 0V e 5 V, ottenuti collegando la porta allo stesso enable del circuito precedente in \fig{enable}. I valore di potenziale  0V e 5V sono interpretati come stati LOW e HIGH da arduinio.\\ 
Sono stati utilizzati cinque stati corrispondenti a quali del tre led risultano accesi. A ciascuno stato è assegnata una variabile intera: \tab{stati_arduino}.
\begin{table}[h]
	\centering
	\begin{tabular}{cc}		
		 STATO& intero\\
		 \midrule 
		 V & 0\\
		VG & 1\\
		R &  2\\
		OFF & 3\\
		G & 4\\ 
		          
 	\end{tabular}
	\caption{ Stati di ARDUINO }
	\label{t:stati_arduino}
\end{table}
La logica della macchina a stati finiti implementata via software si trova nella \tab{semaforo_arduino}. Per ciascuna transizione tra stati diversi è stabilito un tempo di transizione che può essere eventualmente modificato. 
\begin{table}[h]
	\centering
	\begin{tabular}{ccc}		
		 EN& $STATO^{n}$& $STATO^{n+1}$\\
		 \midrule 
		 T & V &  VG\\
		 F & V & OFF\\
		 T & VG & R\\
		 F & VG & OFF\\
		 T & R & V\\
		 F & R & OFF\\
		 T & OFF & R\\
		 F & OFF & G\\
		 T & G & R\\
		 F & G & OFF\\
		          
 	\end{tabular}
	\caption{ Tabella di verità del semaforo con ARDUINO. I valori logici booleani dell'enable possono essere True o False, in base al potenziale letto dalla porta D8 }
	\label{t:semaforo_arduino}
\end{table}

La macchina qui realizzata è del tipo Moore in quanto, i valori dei tre output (D9, D10, D11) all'ingresso dei LED sono determinati solamente dallo stato della macchina e non da altri input.\\

