\section{Scopo e strumentazione}
Lo scopo dell'esperimento è quello di costruire alcuni circuiti logici sequenziali e osservarne le caratteristiche . Si sono usati nell'esperienza ,oltre all'usuale oscilloscopio e ad arduino, anche due circuit integrati NAND (SN74LS00), uno XOR (SN74LS86),due Dual-latch (SN74LS74) e un contatore a 4-bit (SN74LS93). Abbiamo inoltre utilizzato 4 diodi led , un DIP switch a 4 interrutori e un pulsante a doppio contatto, oltre a 4 resistenze, i cui valori misurati con il tester sono $R_1=319 \pm 4\Omega$ $R_2=329\pm 4 \Omega$ $R_3=324\pm 4 \Omega$ $R_4=330\pm 4 \Omega$.


\section{Flip-Flop D-Latch}
Si è montato il circuito in \fig{D-Latch} utilizzando entrambi gli integrati SN74LS00.Il primo integrato è servito per implementare le 4 porte NAND mentre dell'altro integrato è stata usata una sola porta NAND per realizzare un NOT. Si è prelevato il segnale, da mandare in ingresso a questo circuito, dall'uscita $Y_1$ di arduino (onda quadra). Si è collegato l'ENABLE a massa tramite un interruttore del DIP Switch , in modo da osservare l'uscita del circuito al variare dello stato dell'ENABLE. Il ruolo del NOT è quello di evitare lo stato NOT-ALLOWED ovvero di avere due segnali bassi all'uscita di $U_1a$ e $U_1d$. Dallo schemo circuitale prima indicato , si può notare che se l'ENABLE è basso allora le uscite di $U_1a$ e $U_1d$ sono entrambe alte e quindi entrambe le uscite sono nello stato HOLD ovvero non oscillano ;in particolare Q è alta mentre Qbar è bassa. Qbar è mostrato in \fig{Qbar_enable_basso}, in cui l'offset dell'uscita è $130 \pm10$mV.Tale errore di lettura è dovuto all'ampiezza della traccia sull'oscilloscopio.  \\
Nel caso in cui l'ENABLE è alto le uscite di $U_1a$ e $U_1d$ sono sempre opposte e quindi le uscite sono Q=D e QBar=$\bar{D}$. 
Tale comportamento è confermato dai grafici in \fig{Q} e \fig{Qbar}, che quindi verificano anche la tabella di verità del circuito.\\
Abbiamo poi misurato il ritardo temporale tra segnale in ingresso e in uscita (Q). Tale ritardo corrisponde alla distanza temporale, tra il 50\% dell'incremento massimo di entrambi i segnali, durante la fase di salita. Si è ottenuto $27.0 \pm 0.2$ns .
