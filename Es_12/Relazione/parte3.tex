\section{Shift register con D-Latch}
Abbiamo realizzato il circuito in \fig{shift_register} , collegando a massa i preset tramite il pulsante e l'ingresso DATA a massa tramite il Dip Switch. Come CLOCK si è usata ancora l'uscita $Y_1$ di arduino con una frequenza intorno ai 1 Hz. Quando il preset è nello stato basso le uscite sono tutte alte, mentre appena diventa alto le uscite iniziano a spegnersi o ad accendersi in base all'input DATA: lo shift register acquisisce il valore che assume DATA in corrispondenza del fronte di salita del clock. Ad ogni colpo di clock questo valore viene trasferito ordinatamente dall'uscita Q0 fino a quella Q3.