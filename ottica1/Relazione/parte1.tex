\begin{section}*{Scopo e strumentazione}

L'esperienza, divisa in due parti, è volta dapprima alla misura della lunghezza d'onda della riga gialla del sodio, a partire dall'angolo di diffrazione, quindi viene stimata la costante di Rydberg a partire dalle lunghezze d'onda delle righe dell'idrogeno.\\ La strumentazione include uno spettroscopio a prisma (utilizzato nella prima parte), la cui struttura è mostrata in \emph{Figura 1}, uno spettroscopio a reticolo di diffrazione (\emph{Figura 2}, utilizzato nella seconda), una lampada al cadmio, una lampada al mercurio, una lampada all'idrogeno ed una lampada al sodio.

%\begin{figure}[h]
%\begin{center}
%\includegraphics[width=5cm]{}
%\caption{spettroscopio a prisma}
%\end{center}
%\end{figure}

\end{section}

\begin{section}*{Misura della lunghezza d'onda della riga gialla del sodio}

\begin{subsection}*{Calibrazione dello strumento con la lampada al cadmio}

Abbiamo posto la lampada al cadmio nella posizione indicata in \emph{Figura 1}, in modo tale che la luce investisse completamente la fenditura. Si è tolto quindi il prisma, facendo attenzione a non toccarlo (avvalendoci dell'uso della vite) e abbiamo allineato i due telescopi, ricercando la fenditura e centrandola. Leggendo il goniometro abbiamo così potuto determinare un angolo di riferimento. La misura è stata svolta separatamente dai diversi componenti del gruppo, come sempre anche in seguito per le misure di angoli, ottenendo così tre valori da cui sono state ricavate media e scarto quadratico medio, pari a $\theta_{0} = ??? $. Si hanno infatti tre fonti di errore: l'errore casuale così calcolato e gli errori strumentali dovuti alla risoluzione del goniometro (pari a un primo) e alla larghezza della fenditura.\\ %come si trattano?
Ricollocato il prisma nella sua posizione originaria, abbiamo dunque ruotato lo stesso in modo da formare un angolo di almeno 60 gradi con la normale. Ruotando il telescopio, abbiamo allora ricercato le righe spettrali e, agendo ancora sul prisma, abbiamo raggiunto la posizione di minima deviazione per la riga verde. La posizione del prisma non è più stata variata in seguito. Abbiamo così misurato gli angoli per le righe di emissione blu, azzurra, verde e rossa (\emph{Tabella 1}), realizzando un grafico degli stessi in funzione dell'inverso della lunghezza d'onda, nominale, mostrato in \emph{Figura 3}. Si è svolto così un fit lineare a due parametri, che dà come risultati  \\ %fit

\end{subsection}

\begin{subsection}*{Misura con la lampada al sodio}

Sostituita la lampada al cadmio con la lampada al sodio, abbiamo misurato l'angolo a cui si è osservata la banda gialla, pari a $\theta_{s} = 23 \pm 4$ rad. Utilizzando i parametri determinati nella sezione precedente, abbiamo così potuto calcolare il valore della lunghezza d'onda della riga gialla del sodio $\lambda = 598 \pm 11$ nm.

\end{subsection}

\end{section}


