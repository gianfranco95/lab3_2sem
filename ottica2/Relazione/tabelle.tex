\begin{document}
\begin{table}[h]
	\centering
	\begin{tabular}{*{4}{S[table-figures-exponent = 2]} }
		{ $N_{bande}$} &{$\Delta N_{bande}$} & {$X$ [\SI{}{\micro\meter}]} &{$\Delta X $[\SI{}{\micro\meter}]} \\
		\midrule
		50 & 0 & 90 & 10\\ 
		51 & 1 & 90 &  0\\ 
		56 & 1 & 100 &  0\\ 
		85 & 1 & 150 &  0\\ 
		60 & 1 & 110 &  0\\ 

	\end{tabular}
	\caption{$N_{bande}$ è il numero di bande contate sullo schermo a muro con il relativo errore. X è invece il valore letto sul micrometro. Laddove l'errore è 0, significa che può essere trascurato rispetto agli altri. }
	\label{t:calibrazione_michelson}
\end{table}

\begin{table}[h]
	\centering
	\begin{tabular}{*{4}{S[table-figures-exponent = 2]} }
		{ $N_{bande}$} &{$\Delta N_{bande}$} & {$X$ [\SI{}{\micro\meter}]} &{$\Delta X $[\SI{}{\micro\meter}]} \\
		\midrule
		71 & 3 & 120 & 10\\ 
		60 & 3 & 100 & 10\\ 
		75 & 3 & 120 & 10\\ 
		97 & 1 & 150 &  0\\ 
		79 & 1 & 130 &  0\\ 

	\end{tabular}
	\caption{$N_{bande}$ è il numero di bande contate osservando direttamente il fascio luminoso. X è invece il valore letto sul micrometro. Laddove l'errore è 0, significa che può essere trascurato rispetto agli altri. }
	\label{t:mercurio_michelson}
\end{table}