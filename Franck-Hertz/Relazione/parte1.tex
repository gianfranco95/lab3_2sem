\section{Scopo e strumentazione}

Con questa esperienza abbiamo osservato il comportamento degli elettroni in un tetrodo contenente gas neon, soggetti a un campo elettrico esterno. In particolare abbiamo studiato l'andamento della corrente in uscita in funzione del campo applicato. La strumentazione impiegata acclude, un tetrodo ELWE U8482230 e un sistema di alimentazione e lettura di corrente ELWE, dove i valori delle varie grandezze sono mostrati su un display, un oscilloscopio.\\
In \fig{apparato} è mostrato lo schema circuitale dell'apparato utilizzato. Nel tetrodo, alle cui estremità vi sono un catodo da cui partono generalmente gli elettroni e un anodo, sono presenti infatti due griglie; tra la griglia più vicina al catodo e il catodo stesso vi è il potenziale 
$U_G$, che incrementa il tasso di emissione degli elettroni. $U_A$ indica invece la differenza di potenziale tra la seconda griglia e il catodo e permette di accelerare gli elettroni. $U_E$ è, diversamente dagli altri, un potenziale frenante ed agisce tra la seconda griglia e l'anodo. Gli elettroni, una volta raggiunto l'anodo, danno luogo ad una debole corrente($\approx$ nA); un amplificatore permette di visualizzarla tramite un fattore di amplificazione variabile. Ai due canali dell'oscilloscopio sono osservati i potenziali $U_A$ e $V_{out}$, quest'ultimo proporzionale a $I_C$.

\section{Comportamento degli elettroni in seguito ad urti con atomi di neon}

Dopo aver avuto cura di porre tutte le manopole a 0 in modo da annullare eventuali offset, abbiamo attivato il sistema di alimentazione ,impostato la tensione applicata ai capi del filamento, che permette l'emissione di elettroni per effetto termoionico, a $U_F = 8.0 \pm 0.5$ V.\\  Abbiamo atteso che il catodo raggiungesse una condizione di equilibrio.

\subsection{Bande del neon}

Abbiamo lentamente variato $U_G$ finché non abbiamo distintamente osservato una banda luminosa tra le due griglie. Ciò è avvenuto per $U_G = 4.3 \pm 0.1$ V. Abbiamo dunque agito anche sulla manopola di $U_A$ osservando a partire da un certo valore di tensione la banda arancione. Continuando ad incrementare la $U_a$ abbiamo potuto individuare dapprima due e poi tre bande ben distinte.  I valori di tensione e corrente corrispondenti alla comparsa di ciascuna banda ottenuti mantenendo costante $U_G$ (al valore prima indicato) e circa nullo il potenziale frenante $U_e$ sono riportati in ordine in \tab{Va_Ic}. In assenza di potenziale frenante non è stata osservata alcuna diminuzione di corrente in corrispondenza della comparsa delle bande: in accordo con il fatto che tutti gli elettroni giungono all'anodo\\

\begin{table}[h]
	\centering
	\begin{tabular}{ *{4}{S[table-figures-exponent = 5]} }
		{$U_a$ [V]} & {$ \Delta U_a$ [V]} & {$I_C$ [nA]} & {$ \Delta I_C$ [nA]}\\
		\midrule
		22.5  &3   &8 &1 \\
		40.5  & 3  &20 &1\\
		59.0  & 3  &86 &1\\
	\end{tabular}
	\caption{Valori di $I_c$ in funzione di $U_a$}
	\label{t:Va_Ic}
\end{table}

In \tab{Va_Ic2} sono invece riportati i dati ottenuti con una tensione frenante pari a $U_E = 6.6 \pm 0.1$ V e senza agire sulla manopola di $U_G$. Con un simile valore di $U_E$ è stato possibile osservare una decrescita della corrente $I_{c}$ al comparire delle bande, evidenziando così il comportamento non classico dell'atomo di neon in seguito agli urti con gli elettroni: questi ultimi infatti eccitano gli atomi di neon a ben definiti livelli energetici.\\

\begin{table}[h]
	\centering
	\begin{tabular}{ *{4}{S[table-figures-exponent = 5]} }
		{$U_a$ [V]} &{$ \Delta U_a$ [V]} &{$I_C$ [nA]} &{$ \Delta I_C$ [nA]}\\
		\midrule
		25.0 & 3  & 0 & 1 \\
		43.5 & 3  & 2 & 1\\
		60.5 & 3 & 14 & 1\\
	\end{tabular}
	\caption{Valori di $I_c$ in funzione di $U_a$}
	\label{t:Va_Ic2}
\end{table}
Errori così elevati al potenziale sono dovuti al fatto che le bande avevano spessore finito non trascurabile e non erano abbastanza nitide da poter determinare con maggiore precisione il potenziale a cui comparivano. Per ciascuna banda i due valori misurati, con $U_E$ differenti, sono compatibili: infatti la comparsa delle bande è dovuta solo ad $U_A$ e non al potenziale frenante, che influisce solo sulla corrente\\
All'aumentare di $U_{g}$ si osservava un aumento della luminosità delle bande e della corrente di collettore. Ciò si attribuisce al fatto che quanto più grande è questo potenziale, tanto maggiore è il numero di elettroni estratti dal filamento per effetto termoionico.
Inoltre si osserva che lo spessore delle bande tra la prima e la terza è crescente. Infatti gli elettroni, appena superata la griglia catodo hanno una distribuzione in energia centrata nel valore di $U_G$ e con una certa larghezza. L'urto è un processo casuale, che quindi redistribuisce le energie con una larghezza maggiore rispetto alla precedente.\\
\subsection{Generatore di rampa, osservazioni qualitative}

Abbiamo attivato un generatore di potenziale dipendente linearmente dal tempo agendo sulla manopola apposita ed abbiamo posto la tensione di accelerazione al suo livello massimo possibile $U_A = 80.0 \pm 0.5$ V. Ponendo l'oscilloscopio in modalità X-Y abbiamo così potuto osservare $I_C$ in funzione di $U_A$, realizzando un grafico riportato in \fig{figura_1}. \\
I tre minimi corrispondono alla comparsa di ciascuna delle tre bande e di tre massimi corrispondenti. Infatti in corrispondenza di ogni minimo il neon sottrae energia agli elettroni. Il secondo massimo ha un valore di corrente di collettore pari a circa il doppio rispetto a quella del primo.\\
Facendo attenzione a non saturare la corrente di collettore ed agendo quindi opportunamente sul guadagno, abbiamo ricercato il comportamento di $I_C(U_A)$ al variare di $U_E$. In particolare, la corrente diminuisce all'aumentare della tensione frenante, mentre i minimi, e soprattutto il terzo minimo, si spostano verso destra. Questo può essere interpretato considerando il fatto che i minimi sono presenti quando gli elettroni, dopo l'urto acquisiscono nuovamente abbastanza energia per superare la barriera frenante. Di conseguenza quanto più la barriera è elevata tanto più gli elettroni necessitano di una maggiore energia dopo l'urto per superarla.\\ 
Per $U_E$ grande il terzo e il secondo minimo diventano negativi. Ciò è attribuito ad una corrente in senso opposto di elettroni estratti dall'anodo e raccolti nella griglia più vicina ad esso.\\
Abbiamo ricercato il valore di $U_E = 10.2$ V per cui i minimi si disponessero approssimativamente a $I_C = 0$ nA come in \fig{figura_3}.
