\section{Scopo e strumenti}
Lo scopo di questa esperienza è di osservare il funionamento della porta logica NOT per diversi segnali in ingresso e studiare anche il suo comportamento in regime non ideale.\\
Gli strumenti a disposizione includono:
\begin{itemize}
\item{Integrato SN74LS04}
\item{IC SN74LS244 octal buffer/driver}
\item{arduino nano}
\item{trimmer da 2k$\Omega$, 10k$\Omega$, 100k$\Omega$}
\end{itemize}
Nella prima parte è stato analizzata la porta logica in condizioni statiche, nella seconda invece in condizioni dinamiche.\\

\section{Caratteristiche statiche}
\subsection{Potenziali}
E' stato montato il circuito in \fig{circuito1}, composto dalla resistenza $R_{2}=100 \pm 1 \Omega$; trimmer da 1.98$\pm 0.02 k\Omega$ ed una delle porte NOT contenute nell'integrato SN74LS04.\\
 Agendo sul trimmer "R1" è stato possibile variare il potenziale in ingresso alla porta NOT e ne è stata osservata l'uscita. Il potenziale di alimentazione era $V_{cc}$=4.29 $\pm$ 0.03 V(???? non è possibile perchè all'ingressso del not ci sono 4.30 V! ?????) ed è stato utilizzato sia per l'alimentazione dell'integrato SN74LS04 sia ai capi della serie di $R_2$ con il trimmer "R1". Il potenziale in ingresso alla porta spaziava tra $V_{IN,min}=0.0 \pm 0.1$ mV e $V_{IN,max}=4.30 \pm 0.03$ V. Al variare del potenziale in ingresso è stato realizzato un grafico di $V_{OUT}$ in funzione di $V_{IN}$: \fig{out_in}. Si può dare una stima dei potenziali in ingresso che corrispondono al massimo valore per cui il segnale è interpretato logicamente come "0" ($V_{IL}$) e il minimo per cui è interpretato come "1" ($V_{IH}$). Si è considerato come $V_{IL}$ il massimo potenziale in ingresso per cui l'uscita si attesta sul suo valore massimo ("1"). Invece come $V_{IH}$ il minimo valore in ingresso per cui l'uscita è a potenziale minimo ("0"). I valori stimati sono, come è possibile dedurre dal grafico \fig{out_in}  $V_{IL}=1.0 \pm 0.1 V$ e $V_{IH}=1.5V \pm 0.1$ V.(\tab{tabella})
 E' stato possibile stimare queste due grandezze anche dall'osservazione delle tensioni tramite l'oscilloscopio: variando gradualmente la partizione del trimmer, si osservava che per potenziali in ingresso minori di $V_{IL}$ e maggiori di $V_{IH}$ il segnale in uscita rimaneva costante (ai due valori massimo e minimo). Mentre, variando il trimmer nella banda dei potenziali intermedi c'era una brusca variazione dell'uscita. Nelle zone di transizione il potenziale in ingresso  non viene riconosciuto come segnale logico e il segnale in uscita si stabilizza su un valore intermedio compreso tra i valori alto e basso.\\ 
 I valori di  $V_{IL}$ e $V_{IH}$ stimati sono rispettivamente maggiore e minore di quelli riportati sul datasheet (\tab{tabella}); sotto questo aspetto il nostro integrato ha una funzionalità migliore di quello di riferimento. \\
Come misure di potenziale di uscita $V_{OH}$ e $V_{OL}$ abbiamo considerato rispettivamente il valore massimo ed il valore minimo del segnale in uscita, che si presentavano circa costanti lontano dalla zona di transizione: valori riportati in \tab{tabella}.  Essi sono rispettivamente maggiore e minore del valore tipico indicato sul datasheet.\\
Le misure di potenziale sono state effettuate sia tramite multimetro digitale che tramite oscilloscopio. IL multimetro consente in generale una risoluzione migliore dell'oscilloscopio per misure statiche. In questo caso però l'incertezza dovuta al fatto che la transizione tra due regimi di lavoro diverso non era ben definita predominava sugli errori strumentali.
\subsection{Correnti}
In questa sezione, secondo la convenzione del costruttore dell'integrato, si seguirà la convenzione secondo cui il segno delle correnti è positivo se entranti nella porta NOT.\\
\subsubsection{Correnti in ingresso}
Nello stesso circuito precedente è stato inserito in serie all'ingresso della porta l'amperometro digitale, per misurare la corrente in ingresso al NOT.
Abbiamo realizzato, agendo ancora sul trimmer, un grafico della corrente in funzione del potenziale di ingresso, riportato in \fig{Iin_in}, dove i potenziali sono misurati all'oscilloscopio con errori dati da calibrazione e risoluzione dei cursori.\\
L'andamento è globalmente crescente  ed è  compreso tra un valore di corrente uscente massima di $I_{max}\approx -0.26$ mA e un valore massimo corrispondente ad una piccola corrente entrante $\sim 3 \mu$A. \\
Questo andamento può essere interpretato a partire dal circuito caratteristico del NOT in \fig{porta_not}. La presenza del diodo "D2" impedisce il passaggio di corrente entrante. L'unica corrente che interessa l'ingresso è dunque una corrente uscente che fluisce dapprima attraverso la resistenza da 20 k$\Omega$ e quindi attraverso il diodo "D1". Ci si aspetta che quanto più l'ingresso è basso tanto più è grande la caduta di potenziale ai capi della resistenza da 20K e del diodo e tanto sarà maggiore (in valore assoluto) la corrente che attraversa il diodo "D1" in polarizzazione diretta. In particolare ponendo l'ingresso a $\sim$ 0V, trscurando l'impedenza statica del diodo, la resistenza deve sostenere una caduta di potenziale di 5V, di conseguenza sarà attraversata da una corrente nominale di I=0.25 mA. Questa è la massima corrente che può uscire dall'ingresso della porta, e ciò è in accordo con il valore misurato($\sim$-0.26 mA).\\
I valori delle correnti $I_{IH}$ e $I_{IL}$ sono per definizione le correnti massime erogabili (o assorbibili) dalla porta nei due regimi in ingresso. Quando l'ingresso è basso la corrente massima che viene erogata è -0.25 $\pm$ 0.03 mA; mentre quando l'ingresso è alto la porta assorbe corrente al massimo pari a $3 \pm 1 \mu A$ . Gli errori attribuiti a queste due misure sono quelli dovuti al multimetro. Le due misure sono i valori che si leggono agli estremi dell'intervallo di esplorazione di $V_{in}$. In realtà questo intervallo non coincide esattamente con quello ideale 0-5V, ma essendo le misure delle correnti affette da un notevole errore strumentale, si può dire che i valori trovati per $I_{IH}$ e $I_{IL}$ sono una buona stima entro l'errore.
\subsubsection{Correnti in uscita}
Per misurare $I_{OL}$ ed $I_{OH}$ abbiamo montato il circuito mostrato in \fig{circuito2}, dove $R_2$ è quella già utilizzata in precedenza ed il trimmer ha resistenza massima $R = 102.2 \pm 0.8$ k$\Omega$.\\
Si è operato come segue:
\begin{itemize}
\item per misurare $I_{OL}$ abbiamo collegato a $V_{cc}$ tanto il potenziale di ingresso quanto il potenziale misurato al punto A. 
Se la resistenza tra uscita ed A è massima l'uscita si mantiene bassa. Diminuendo questa resistenza la corrente che fluisce attraverso la porta tende ad aumentare; in prossimità del valore massimo $I_{OL}$ per evitare l'eccessivo passaggio di corrente l'uscita $V_{OUT}$ aumenta.\\
 Abbiamo agito sul trimmer fino a che l'uscita, visualizzata all'oscilloscopio, non cambiava stato passando al livello alto. C'era una notevole ambiguità sulla posizione limite del trimmer per cui l'uscita si manteneva bassa.Abbiamo proceduto quindi in due modi: dapprima misurando la corrente di uscita per la posizione oltre la quale $V_{out}$ non variava lentamente ma saltava quasi istantaneamente al livello superiore; nel secondo modo valutando la massima corrente per cui $V_{OUT}$ si manteneva compatibile con $V_{OL}$. Nel primo caso abbiamo ottenuto $I_{OL} = 824 \pm 12$ $\mu$A per una tensione $V_{OUT}=150 \pm 6$ mV, mentre per il secondo (i cui risultati sono riportati in \tab{tabella}) $I_{OL} = 147 \pm 4$ $\mu$A .
\item Per misurare $I_{OH}$ abbiamo invece collegato i potenziali in ingresso e in A a terra. In tal caso se la resistenza è massima ho un potenziale massimo in uscita ed analogamente alla situazione precedente, diminuendo la resistenza  $V_{out}$ tende a diminuire a causa dell'eccessiva richiesta di corrente. \\
Variando il trimmer abbiamo misurato, come in precedenza, la corrente $I_{OH}$ in due modi: il primo osservando la posizione del trimmer per cui il potenziale in uscita restava alto subito prima di decrescere entrando così in regime non definito, ottenendo così $-5.5\pm 2 \mu A$. L'incertezza è dovuta al fatto che il passaggio al regime non ideale era non ben evidente. In secondo luogo è stata valutata la massima corrente che fluiva quando $V_{OUT}$ era compatibile con $V_{OH}$, il risultato è  $I_{OH}$=-37 $\pm$1 $\mu A$(anche in \tab{tabella} ).\\
Queste correnti di uscita sono state acquisite indirettamente, misurando il potenziale ai capi della resistenza $R_{2}$.
\end{itemize}
In \tab{tabella} sono riportate le grandezze misurate.\\
\begin{table}[h]
	\centering
	\begin{tabular}{ccc}
		{} & {Valore misurato} & {datasheet} \\
		\hline
    $V_{IH}$ [V] 	  & 1.5$\pm$0.1&  2(min) \\
    $V_{IL}$ [V]	  & 1.0$\pm$0.1&  0.8(max)  \\
    $V_{OH}$ [V]	  & 4.1$\pm$0.1&  2.7(min)\\ %& min(typ)\\ %
    $V_{OL}$ [mV]     & 114 $\pm$4&   400(max) \\%& max{Iol=4mA}\\
    $I_{IH}$ [$\mu$A] & 3 $\pm$1&     20(max)  \\
    $I_{IL}$ [mA]     & -0.25$\pm$0.03&  -0.4(max) \\
    $I_{OH}$ [$\mu$A] & -37 $\pm$ 1&  -400(max)  \\
    $I_{OL}$ [$\mu$A] & 147 $\pm$ 4&  8000(max) \\
	
  
 	\end{tabular}
	\caption{ Le misure di potenziale sono state effettuate tramite oscilloscopio. Quelle di corrente di ingresso tramite amperometro digitale (multimetro) posto tra il trimmer e l'ingresso della porta. Le correnti di uscita invece indirettamente misurando la tensione ai capi della resistenza $R_{2}$ . Gli errori per i potenziali in ingresso sono sensibilmente maggiori delle risoluzione degli strumenti utilizzati, in quanto non era definita nettamente la transizione tra due regimi di lavoro. Per i potenziali in uscita invece sono dovuti alla risoluzione degli strumenti come somma in quadratura di errore di calibrazione  e incertezza di lettura}
	\label{t:tabella}
\end{table}

\subsection{confronto con datasheet}
Dal confronto con i dati del datasheet si può dedurre che, per quanto riguarda i valori di potenziale, questi garantiscono una funzionalità migliore della porta. Infatti in ingresso le bande corrispondenti ai segnali "0" e "1" sono più ampie e i valori di potenziale alto e basso sono rispettivamente maggiore e minore rispetto a quelli indicati dal costruttore.\\
Anche per quanto riguarda le correnti in ingresso, esse sono minori rispetto ai valori massimi indicati nel datasheet. Questo può essere considerato un pregio delle porte perché quanto minore è la corrente che assorbono, tanto è maggiore il numero di porte di questo tipo che possono essere pilotate dall'uscita di un circuito.\\
Invece le correnti in uscita, risultate minori di quelle previste, determinano una minore capacità per il NOT di sostenere altre porte.\\
In generale il fanout dipende dalla porta logica, dalla cui uscita, si preleva l'ingresso per alimentare altre porte logiche generalmente diverse dalla precedente. L'unica cosa che si può valutare nel nostro caso è il rapporto $\frac{I_{OL}}{I_{IH}}= 49$ che indica il numero di porte NOT che possono essere comandate da un'unica altra porta NOT.