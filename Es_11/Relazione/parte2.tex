\section{Multivibratore Astabile}
E' stato montato il circuito in \fig{astabile} ed è stata misurata la forma d'onda in uscita che risultava essere quadra come mostrato in \fig{astabile_out_vc}. Ne è stato misurato il periodo $T= 189 \pm 1 \mu s$ e duty cycle del  $65 \pm 1$\%. \\
E' stata visualizzato all'osciloscopio l'ingresso del primo NAND (\fig{astabile_out_vc}) da cui si possono notare due andamenti caratteristici della carica e scarica del condensatore. L'andamento decrescente dell'ingresso corrispondente all'uscita alta, rappresenta la carica del condensatore quando l'ingresso della NAND 7 è alto, la sua uscita bassa e l'ingresso di NAND 5 è alto. La carica del condensatore \footnote{La corrente fluisce dall'ingresso della NAND 7 verso il suo ingresso passando per condensatore e resistenza. } determina un abbassamento del segnale all'ingresso di NAND 5.In questa maniera tale ingresso commuta verso lo stato basso e gli stati precedenti relativi a tutte le porte sono invertiti.Al momento della commutazone,essendo l'uscita della NAND 6 a zero e il condensatore ancora carico, allora $V_{C2}$ risultra negativo . Da questo momento inizia a fluire corrente in verso opposto al precedente determinando inizialmente la scarica del condensatore e poi una carica di segno opposto a quella iniziale.Nel momento in cui il condenstaore raggiunge un certo valore di carica sufficiente a superare $V_{IH}$ della NAND 5 allora si ha una commutazione di tutti gli stati e inizia una nuova oscillazione del segnale. Per carica e scarica del condensatore ci si è riferiti ad una particolare convenzione dsul segno delle correnti.\\
E' stata cambiata la resistenza $R_2$ e si sono misuarati i corrispettivi periodi delle onde quadre in uscita. I risultati sono in \tab{} e sono ache riportati in \fig{periodo}. Si può notare l'andamento lineare del periodo in funzione della resistenza. E' stato realizzato un fit lineare al modello y=mx+q ottenendo: m=180 $\pm$ 3 ns/$\Omega$ ; q=13 $\pm$2 $\mu$s; $\chi^{2}$/ndof=0.8/4.\\
\begin{table}[h]
	\centering
	\begin{tabular}{ccc}
		{R [$\Omega$]} & {T [$\mu $s]} & {$\Delta$T [$\mu $s]} \\
		\midrule
           456 & 94.4 & 0.4\\
           560 & 114 & 1\\
           675 & 135 & 1\\
           985 & 189 & 1\\
           1173 & 222 & 1\\
           1507 & 284 & 1\\
 	\end{tabular}
	\caption{ Periodo T in uscita in funzione della resistenza R, con errore di lettura (errori sulle resistenze dovuti al tester digitale)}
	\label{t:periodo}
\end{table}
\section{Generatore di onda quadra}
Sono stati utilizzati i due circuiti astabile e monostabile, montati in precedenza, per realizzare un generatore di onda quadra: \fig{generatore_circuito}. Il circuito è ottenuto semplicemente ponendo il segnale in uscita al multivibratore astabile all'ingresso di un passa-alto. L'uscita di quest'ultimo è stata immessa all'ingresso del multivibratore monostabile. \\
Assumendo un funzionamento ideale, in uscita OUT-A si ha un'onda quadra con periodo dipendente da $R_{2}$ come analizzato in precedenza per l'astabile. Il derivatore trasforma quest'onda in due impulsi: uno positivo e un altro negativo all'uscita IN-M. Il circuito monostabile è sensibile solamente all'impulso positivo che è quindi visto come trigger.
 All'uscita OUT-M si osserva dunque un'onda quadra il cui fronte massimo è triggerato dal segnale in IN-M ed ha una durata che dipende solamente da $R_{1}$ e non dal trigger. Il periodo, invece, dato che il monostabile è azionato dall'impulso positivo in IN-M, dipende quindi solamente dalla frequenza dell'astabile e quindi solo da $R_{2}$.\\
 E' stato verificato ciò effettuando misure di periodo e duty cicle al variare delle resistenze $R_{1}$ ed $R_{2}$ riportate in \tab{}. Si può notare dai dati che effettivamente $R_{2}$ influisce solo sul periodo, e $R_{1}$ solo sulla durata del fronte massimo in uscita OUT-M e quindi sul duty cicle. \\
 Inoltre, come atteso, i valori di periodo e di durata del fronte alto sono compatibili rispettivamente con quelli misurati in precedenza con le stesse resistenze per i singoli circuiti astabile e monostabile separatamente.\\
 In realtà si è potuto osservare che il comportamento del generatore di onda quadra non è perfettemente ideale: lo sarebbe se ciascun componente avesse impedenza in ingresso infinita e in uscita nulla. Infatti ci si aspetterebbe che il circuito passa alto fornisca due impulsi e si stabilizzi ad un valore nullo successivamente. In realtà si osserva che il valore asintotico del potenziale presenta un offset di ???V. Questo è stato attribuito al fatto che il monostabile presenta un impedenza in ingresso non sufficientemente elevata per il valore di resistenza $R_{3}$. In effetti quanto più la resistenza $R_{3}$ è grande tanto più l'offset aumenta (\tab{} ), in accordo con il fatto che lasciando IN-M flottante l'ingrsso della NAND1 assuma il valore alto. Per valore di resistenza $R_{3} \sim 1 k\Omega$ l'offset e sufficientemente elevato da non permettere alla NAND2 di riconoscere come valore in ingresso come basso in assenza di impulsi. Quindi essa,nell'intervallo tra due impulsi, permane in regime di transizione . Cambiando $R_{3}$ con valori più bassi si osserva che le porte raggiungono i loro rgimi ideali(alto, basso). In ogni caso per le resistenze utilizzate, all'uscita OUT-M non si osservano significative deformazioni della forma d'onda quadra, nonostante i regimi non ideali delle porte.\\
 Infine, tenendo conto delle misure effettuate in precedenza per i singoli circuiti, si sono scelte le resistenze $R_{1}=385 \pm ??$ ed $R_{2}=456 \pm ???$ in modo tale da ottenere un periodo T=$95 \pm 1 \mu s$ ed un duty cicle di 33 $\pm$ 1 \%.
 
 \section{Conclusioni}
 E' stato analizzato il funzionamento della porta nand utilizzata per la costruzione di circuiti composti. Il comportamento è in accordo con quello atteso.
