\section{divisore di frequenza}
E' stato realizzato il circuito in \fig{contatore}, composto da 4 flip-flop jk (contenuti nell'integrato 74LS93). Tutti i flip flop hanno all'ingresso J il segnale in uscita dalla porta NAND1 e l'uscita Q di ognuno funge da clock per il flip-flop successivo.
Per visualizzare i valori delle uscite dei flip-flop nel tempo, è stata collegata ciascuna uscita alla terra tramite un diodo led (per osservare tramite l'accensione il passaggio di corrente ) e una resistenza $\sim 330 \Omega$ (per limitare la corrente). Perchè il circuito funzionasse è stato necessario porre almeno uno dei due ingressi R0 e R1 del NAND1 a massa, lasciando l'altro eventualmente flottante.\\
E' stato inviato, tramite arduino, un segnale di clock di frequenza $\sim 1 Hz$, in particolare $f = 1.02 \pm 0.01$ Hz, ricavato a partire dalla misura del periodo all'oscilloscopio con i cursori, con errore dato dalla parte di lettura. Si è osservato che le 4 uscite Q dei flip flop assumono valori corrispondenti, in logica binaria, ai numeri da 0 a 15 in maniera ordinata e crescente. Nel circuito (in \fig{contatore} ) le uscite sono ordinate in modo tale che l'uscita del FF4 rappresenti la cifra più significativa, mentre  quella del FF1 quella meno significativa.\\
Sempre con arduino è stato inviato un segnale di frequenza più elevata $\sim$ 50 kHz ($f = 50.5 \pm 0.5$ kHz ricavato ancora dalla misura del periodo all'oscilloscopio ). Sono state visualizzate le uscite all'oscilloscopio. E' stato osservato così che la frequenza dell'uscita Q1 del primo FF1 è la metà di quella del clock(\fig{Qa_vs_clock}), mentre per il flip-flop successivi ciascuna uscita (Q2, Q3, Q4 da sinistra a destra) ha una frequenza dimezzata rispetto a quella precedente(\fig{Qb_vs_clock}; \fig{Qc_vs_clock}; \fig{Qd_vs_clock}).\\
E' stato misurato per ciascuna uscita il ritardo del segnale rispetto a quello di clock. La misura è stata effettuata valutando l'intervallo temporale che intercorreva tra gli istanti in cui i due segnali raggiungevano la metà del loro valore massimo, in corrispondenza del loro fronte di discesa. In corrispondenza del fronte di salita del clock l'uscita ha un fronte di discesa; il tempo di ritardo, misurato all'oscilloscopio, risultava circa lo stesso di quello corrispondente al fronte di discesa del clock (comportamento simmetrico).\\

\begin{table}[h]
	\centering
	\begin{tabular}{ccc }
		
		 & Ritardo [ns]& errore\\
		 \midrule 
		  $Q_{1}$ & 14.4 & 0.4 \\
		  $Q_{2}$ & 26 & 1 \\
		  $Q_{3}$ & 38 & 1 \\
          $Q_{4}$ & 50 & 1 \\
 	\end{tabular}
	\caption{ Ritardi dei segnali in uscita }
	\label{t:Ritardi}
\end{table}
Dai dati si può osservare che ciascun flip flop determina un ritardo di $\sim$ 12 ns rispetto al suo ingresso \fig{ritardo}.\\
Infine si è modificato il circuito in modo tale che le uscite si azzerassero una volta raggiunto il numero 10. E' stato realizzato un reset asincrono, cioè che non era azionato dal clock, ma che agiva in base agli output dei flip flop. Sono state quindi prelevate le uscite Q2 e Q4 ed inviate all'ingresso di una porta AND. L'uscita di quest'ultima è stata inviata su uno dei reset R0 o R1, lasciando l'altro flottante. In questo modo non appena veniva raggiunto il 10 Q2 e Q4 assumevano il valore alto portando alta l'uscita dell'AND (che in tutti gli altri casi è bassa),che veniva portata all'ingresso della NAND1 , la cui uscita diventava bassa. Così facendo gli ingressi J dei flip flop JK erano bassi mentre quelli K, flottanti ,alti e quindi venivano tutti resettati . Sono visualizzati così i numeri che vanno da 0 a 9. Il periodo in Q4, misurato all'oscilloscopio, è il decuplo di quello di clock ($640 \pm 4$ ms per un ingresso di $64.0 \pm 0.4$ ms).\\
La porta AND è stata realizzata, come per il flip-flop D-latch, con due NAND dell'integrato SN74LS00: i due ingressi sono stati inviati agli ingressi del primo NAND e la sua uscita all'ingresso del secondo NAND, lasciando l'altra flottante. L'uscita del secondo NAND rappresenta l'uscita dell'AND.