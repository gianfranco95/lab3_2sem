\section{divisore di frequenza}
E' stato realizzato il circuito in \fig{contatore}, composto da 4 flip-flop jk (contenuti nell'integrato 74LS93). Tutti i flip flop hanno all'ingresso il segnale in uscita dalla porta NAND1 e l'uscita di ognuno funge da clock per il flip-flop successivo.
Per visualizzare i valori delle uscite dei flip-flop nel tempo, � stata collegata ciascuna uscita alla terra tramite un diodo led (per osservare tramite l'accensione il passaggio di corrente ) e una resistenza $\sim 330 \Omega$ (per limitare la corrente). Perch� il circuito funzionasse � stato necesssario porre almeno uno dei due ingressi R0 e R1 del NAND1 a massa.\\
E' stato inviato, tramite arduino, un segnale di clock di frequenza $\sim 1 Hz$( periodo di (980 $\pm$ 10) ms, con errore ????????????attribuito). Si � osservato che le uscite Q dei flip flop assumono valori corrispondenti ai numeri da 0 a 15 in maniera ordinata e crescente. Nel circuito (in \fig{contatore} ) le uscite sono ordinate in modo tale che l'uscita del FF1 rappresenti la cifra pi� significativa, mentre  quella del FF4 quella meno significativa.\\
Sempre con arduino � stato inviato un segnale di frequenza pi� elevata $\sim$ 50 kHz (periodo di 19.8 $\pm$ 0.2 $\mu$s ?????errore ). Sono state visualizzate le uscite all'oscilloscopio. E' stato osservato cos� che la frequenza dell'uscita del primo FF1 � il doppio di quella del clock(\fig{Qa_vs_clock}), mentre per il flip-flop successivi ciascuna uscita ha una frequenza doppia di quella precedente(\fig{Qb_vs_clock}; \fig{Qc_vs_clock}; \fig{Qd_vs_clock}).\\
E' stato misurato per ciascuna uscita il ritardo del segnale rispetto a quello di clock. La misura � stata effettuata valutando l'intervallo temporale che intercorreva tra gli istanti in cui i due segnali raggiungevano la met� del loro valore massimo, in corrispondenza del loro fronte di salita.\\

\begin{table}[h]
	\centering
	\begin{tabular}{ccc }
		
		 & Ritardo [ns]& errore\\
		 \midrule 
		  $Q_{A}$ & 14.4 & 0.4 \\
		  $Q_{B}$ & 26 & 1 \\
		  $Q_{C}$ & 38 & 1 \\
          $Q_{D}$ & 50 & 1 \\
 	\end{tabular}
	\caption{ Ritardi dei segnali in uscita }
	\label{t:Ritardi}
\end{table}
Dai dati si pu� osservare che ciascun flip flop determina un ritardo di $\sim$ 12 ns rispetto al suo ingresso.\\
Infine si � modificato il circuito in modo tale che le uscite si azzerassero una volta raggiunto il numero 10. E' stato realizzato un reset asincrono, cio� che non era azionato dal clock, ma che agiva in base agli output dei flip flop. Sono state quindi prelevate le uscite Q1 e Q3 ed inviate all'ingresso di una porta and.L'uscita di quest'ultima � stata inviata su uno dei reset R0 o R1, lasciando l'altro flottante. In questo modo non appena veniva raggiundo il 10 Q1 e Q3 assumevano il valore alto portando alta l'uscita dell'AND (che in tutti gli altri casi � bassa), resettando cos� il contatore.\\
La porta AND � stat realizzata con due NAND dell'integrato SN74LS00: i due ingressi sono stati inviati agli ingressi del primo NAND e la sua uscita all'ingresso del secondo nand, lasciando l'altra flottante. L'uscita del secondo NAND rappresenta l'uscita dell'AND.


