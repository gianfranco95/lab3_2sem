\section{Elettroni in campo magnetico}
Abbiamo inserito il bulbo di vetro, collocando la macchina fotografica in una posizione tale da vedere il piano su cui giacevano gli elettroni (perpendicolare all'asse delle bobine) ortogonale alla congiungente obiettivo-occhio. Abbiamo cercato inoltre di visualizzare la scala graduata in modo che passasse dal centro del bulbo. La posizione della macchina fotografica è stata fissata al pavimento adoperando del nastro adesivo. Per evitare differenze di calibrazione tra le varie immagini è opportuno infatti scattare le foto alla stessa distanza, con lo stesso ingrandimento e la stessa modalità.\\
Svolte le operazioni indicate, che includono la messa a fuoco e la copertura delle bobine con un panno al fine di visualizzare meglio la traccia luminosa, abbiamo variato V$_{heat}$, scattando una foto per ciascun valore di 4V, 5V, 6V, 7V, riportate in \fig{figura_3}, \fig{figura_4}, \fig{figura_5}, \fig{figura_6} rispettivamente. Abbiamo osservato come il raggio tenda ad aumentare per valori di V$_{heat}$ crescenti, mentre per valori più piccoli si evidenzia anche una struttura ad elica e a V$_{heat} = 3$ V il segnale è a malapena visibile. Tali figure sono state ottenute tutte con le stesse V$_{acc} = 246.1$ V e I$_{coil} = 1.18$ A. Mantenendo invece V$_{heat} = 6$ V abbiamo variato V$_{acc}$ e I$_{coil}$ al fine di ricavare il raggio.\\
\subsection{Calibrazione}
Per poter svolgere la calibrazione è invece necessario, in assenza di bulbo, fotografare la scala graduata (con il dispositivo di illuminazione acceso) e prendere le posizioni delle tacchette; per ciascuna di esse ci sarà quindi un valore in pixel (in componenti x ed y, definiti come x e come y rispettivamente), dalla digitalizzazione della foto, ed un valore in centimetri visualizzabile sul righello, definito come z. L'obiettivo di questa sezione è di determinare il coefficiente di calibrazione q [cm/pixel] che associa al pixel la lunghezza in centimetri.\\  
In primo luogo si nota dalle fotografie, che la macchina fotografica non era perfettamente orizzontale. Di conseguenza nelle foto il righello compare obliquo. Per questo motivo calibrare considerando solo le ascisse x in pixel sarebbe equivalente a considerare non le lunghezze reali in centimetri ma la proiezione di queste sull'asse x della fotocamera. Questo corrisponderebbe a cambiare q di un fattore pari al coseno dell'angolo di inclinazione della fotocamera.\\
Per questo motivo, stabilito il riferimento, nello zero del righello, sono stati acquisiti i punti relativi a ciascuna tacchetta in modo essi fossero su una retta parallela al righello: ad ogni valore z in centimetri è quindi associato il numero di pixel $\sqrt{x^2 +y^2}$.\\ 
Si vede facilmente che per calibrare basta: 
\begin{itemize}
\item{eseguire il fit al modello x=ay}
\item{eseguire il fit al modello z=by}
\item{ricavare\footnote{il sistema di equazioni formato dalle due rette può essere visto come l'equazione di una retta nello spazio (x,y,z). ad una misura in centimetri z, la retta fa corrispondere una coppia (x,y) tale che $\sqrt{x^2+y^2}$ è la distanza del punto dall'origine in pixel} 1/q=$\alpha = \frac{\sqrt{x^2+y^2}}{z} = \frac{ \sqrt{1+a^2}}{b}$}
\end{itemize}
Dai fit si ottiene: $a = 73.4 \pm 0.1 $ e $b = 0.47 \pm 0.07$ cm/pixel; $ \alpha = 158 \pm 23 \frac{pixel}{cm}$.\\ 
I fit sono stati eseguiti considerando gli errori sulle ordinate e quelli sulle ascisse dovuti alla discretizzazione dei pixel e alla nitidezza dell'immagine, opportunamente propagati. Abbiamo stimato errori di 3 pixel per le y e di 4 pixel per le x, quindi confrontabili tra loro, mentre l'errore sulle z è stato considerato nullo.
\subsection{Misura dei raggi}
Abbiamo scattato numerose foto del dispositivo con V$_{heat} = 6$ V, prendendo numerosi punti (nell'ordine del centinaio per ciascuna foto) lungo la traccia (ove visibile), approssimativamente a metà della stessa onde minimizzare la parte statistica dell'errore dovuto allo spessore. Per i set di coordinate relativi a ciascuna foto abbiamo allora svolto un fit analitico circolare.\\
 Per tenere conto della geometria proiettiva abbiamo misurato la distanza tra righello ed obiettivo $d=64.9 \pm 0.1$ cm e quella tra obiettivo e centro del bulbo $l=57.2 \pm 0.2$ cm, con l'ausilio del metro. Sono stati modificate quindi le distanze ricavate dalla calibrazione moltiplicandole per un fattore $l/d$ che tiene conto del fatto che la circonferenza luminosa nella foto è proiettata sul piano del righello.\\
 La presenza del bulbo di vetro non portava ad una distorsione significativa della traccia del fascio elettronico, per rifrazione della luce. Si è potuto valutare ciò dal fatto che se ci fosse stata una tale deformazione, le circonferenze, non essendo centrate nello stesso centro del bulbo, avrebbero risentito di una distorsione, che non è stata osservata\footnote{Si è cercato di mantenere le traiettorie non eccessivamente a ridosso della superficie del bulbo}. Inoltre è stata effettuata una calibrazione pixel-centimetri interponendo il bulbo tra la fotocamera e il righello, ma non è stata osservata alcuna differenza rispetto alla calibrazione senza bulbo.\\
Nel fit circolare, analitico, sono stati utilizzati i punti espressi in pixel ed abbiamo considerato come errore una media degli errori sulle x e sulle y. Non sono stati utilizzati i valori già in centimetri in quanto le incertezze sui fattori moltiplicativi di calibrazione  portavano ad avere errori sulla misura in centimetri dilatati ed in maniera non uniforme in base alla distanza dei punti dall'origine, nonostante l'incertezza sui pixel fosse uniforme. Questo processo portava ad una sovrastima dell'errore sul raggio ricavato dal fit. Pertanto si è ritenuto opportuno eseguire il fit sui dati in pixel e, ricavato il raggio in pixel, convertirlo in centimetri.
 Per un'immagine sono stati campionati i punti sia lungo il limite esterno che lungo il limite interno della traccia, prendendo poi x e y come una media su due valori e propagando gli errori di conseguenza. Il fit è a tre parametri;le coordinate del centro e il raggio. L'operazione di fit viene ripetuta per ciascuna delle 29 foto realizzate ottenendo così 29 diversi valori di raggio. Assieme alla tensione di accelerazione e alla corrente corrispondenti, per cui l'errore è sull'ultima cifra significativa, i risultati sono riportati in \tab{tabella2}. Noto B$_{zmax}[T] = 7.80*10^(-4)I_{coil}[A]$ ed $\frac{e}{m}= \frac{2V_{acc}}{{B_{z}r}}^2$ abbiamo anche riportato il valore di e/m.\\
\begin{table}[h]
	\centering
	\begin{tabular}{*{6}{S[table-figures-exponent = 2]} }
		{$V_{acc}$[V]} & {$I_{coil}$[A]} & {$r$[cm]} & {$\Delta r$[cm]} & {$e/m$[V/C]} & {$\Delta e/m$[V/C]} \\
		\midrule
		293.1 & 1.24 & 6.3 & 0.9 & 1.6 & 0.5\\ 
		279.9 & 1.24 & 6.0 & 0.9 & 1.7 & 0.5\\ 
		279.9 & 1.34 & 5.6 & 0.8 & 1.6 & 0.5\\ 
		279.9 & 1.55 & 4.9 & 0.7 & 1.6 & 0.5\\ 
		279.9 & 1.87 & 3.9 & 0.6 & 1.7 & 0.5\\ 
		237.0 & 1.87 & 2.8 & 0.4 & 2.9 & 0.8\\ 
		237.0 & 1.43 & 4.7 & 0.7 & 1.7 & 0.5\\ 
		237.0 & 1.29 & 5.2 & 0.8 & 1.7 & 0.5\\ 
		237.0 & 1.58 & 4.0 & 0.6 & 2.0 & 0.6\\ 
		252.3 & 1.23 & 5.7 & 0.8 & 1.7 & 0.5\\ 
		252.3 & 1.44 & 5.0 & 0.7 & 1.6 & 0.5\\ 
		252.3 & 1.61 & 4.3 & 0.6 & 1.7 & 0.5\\ 
		252.3 & 1.76 & 3.7 & 0.5 & 2.0 & 0.6\\ 
		265.2 & 1.23 & 5.8 & 0.9 & 1.7 & 0.5\\ 
		265.2 & 1.40 & 5.2 & 0.8 & 1.7 & 0.5\\ 
		265.2 & 1.53 & 4.8 & 0.7 & 1.6 & 0.5\\ 
		265.2 & 1.70 & 4.2 & 0.6 & 1.7 & 0.5\\ 
		271.2 & 1.70 & 4.3 & 0.6 & 1.7 & 0.5\\ 
		271.2 & 1.54 & 4.8 & 0.7 & 1.6 & 0.5\\ 
		271.2 & 1.39 & 5.3 & 0.8 & 1.6 & 0.5\\ 
		271.2 & 1.23 & 6.1 & 0.9 & 1.6 & 0.5\\ 
		292.9 & 1.31 & 5.9 & 0.9 & 1.6 & 0.5\\ 
		292.9 & 1.44 & 5.4 & 0.8 & 1.6 & 0.5\\ 
		292.9 & 1.56 & 5.1 & 0.7 & 1.6 & 0.5\\ 
		292.9 & 1.69 & 4.7 & 0.7 & 1.5 & 0.5\\ 
		226.0 & 1.18 & 5.2 & 0.8 & 2.0 & 0.6\\ 
		226.0 & 1.30 & 5.0 & 0.7 & 1.8 & 0.5\\ 
		226.0 & 1.40 & 4.5 & 0.7 & 1.9 & 0.6\\ 
		226.0 & 1.61 & 4.5 & 0.7 & 1.4 & 0.4\\ 
		246.0 & 1.56 & 4.3 & 0.6 & 1.8 & 0.5\\ 
		246.0 & 1.24 & 5.5 & 0.8 & 1.7 & 0.5\\ 
		246.0 & 1.18 & 5.8 & 0.8 & 1.7 & 0.5\\ 
		246.1 & 1.18 & 5.3 & 0.8 & 2.0 & 0.6\\ 
		246.1 & 1.18 & 5.6 & 0.8 & 1.8 & 0.5\\ 
		246.1 & 1.18 & 5.7 & 0.8 & 1.8 & 0.5\\ 
		246.1 & 1.18 & 5.8 & 0.9 & 1.7 & 0.5\\ 
		246.1 & 1.18 & 2.3 & 0.3 & 10.8 & 3.2\\ 
		207.0 & 1.30 & 4.3 & 0.6 & 2.2 & 0.6\\ 
		206.9 & 1.17 & 5.0 & 0.7 & 2.0 & 0.6\\ 
		206.9 & 1.07 & 5.3 & 0.8 & 2.1 & 0.6\\ 
		265.2 & 1.47 & 4.8 & 0.7 & 1.7 & 0.5\\ 
		265.2 & 1.47 & 5.2 & 0.8 & 1.5 & 0.4\\ 
		279.9 & 1.55 & 4.9 & 0.7 & 1.6 & 0.5\\ 
  
 	\end{tabular}
	\caption{ }
	\label{t:tabella2}
\end{table}
Dai risultati abbiamo potuto ricavare una media pesata dei valori di e/m, pari a $\frac{e}{m} = 2.8 \pm 0.1 \frac{10^{11} C}{Kg}$ . Essa non risulta in accordo con il valore atteso $\frac{e}{m} = 1.7589 10^{11} \pm 0.0002 \frac{C}{Kg}$ .\\
Si ritiene che la misura di $\frac{e}{m}$ effettuata oltre all'errore dià indicato, sia affetta da un errore sistematico. E' presente una manopola "Focus" la cui funzione dovrebbe essere solamente quella di focalizzare il fascio. Si è notato invece che agendo su di essa cambiava il raggio del fascio di elettroni, non riuscendo però a misurare variazioni della corrente nelle bobine e del potenziale acceleratore. Sono state catturate foto corrispondenti a $V_acc$ e $I_coil$ fissi e raggio del fascio massimo e minimo al variare del Focus. Sono state analizzate le immagini come nelle sezioni precedenti e sono stati ricavati i due valori di $\frac{e}{m}$ corrispondenti. Nel caso particolare di $V_acc =265.2 \pm 0.1 V$ e $I_coil=1.47 \pm 0.01 A$  si sono ottenuti i due valori: $\frac{e}{m}=2.5 \pm 0.7 10^{11} C/Kg$ e $\frac{e}{m}=2.8 \pm 0.8 10^{11} C/Kg$. In generale si ritien quindi che la misura sia affetta da un 10\% sistematico.\\
Come si può notare dalla fiugra \fig{mappatura} essendo i raggi delle circonferenze misurate al massimo di circa 5 cm, il campo magnetico a cui sono sottosposti gli elettroni è in generale da quello al centro al massimo del 2\%. Dato che la traiettoria non è nemmeno centrata sull'asse di simmetria delle bobine, gli elettroni in realtà vengono deflessi da un campo che è leggermente variabile. Non è immediato tener conto di questo dettaglio, ma dato che in e/m il campo B compare in modo quadratico esso può influire al massimo al 4\% sull'errore della misura. Quindi si ritiene che l'incertezza principale non sia dovuta a queto effetto. \\
Inoltre ci si aspetta che il potenziale $V_heat$, abbia la sola funzione di variare il numero di elettroni emessi per effetto termoionico. Ma invece si osserva che questo influisce sui raggi di curvatura degli elettroni.\\
Infine non si ritiene che il disaccordo tra valore misurato e quello atteso sia dovuto ad una errata correzione di giometria proiettiva. Infatti senza aggiungere questo fattore il valore di e/m misurato sarebbe minore e comunque sovrastimato.
Abbiamo realizzato anche un grafico di e/m in funzione di V$_{acc}$, riportato in \fig{figura_8}, mentre in \fig{figura_9} è rappresentato un istogramma diviso in barre spesse 0.1 , con il numero di occorrenze di e/m per ogni barra sulle ordinate. Abbiamo infine riportato in \fig{figura_10} un grafico di ${(B_z r)}^2$in funzione di V$_{acc}$.

\section{Conclusioni}
Le misure di campo magnetico in funzione della posizione tramite la sonda sono in accordo con quanto atteso.
In generale le misure sono affette principalmente da errori dovuti alla risoluzione del metro a nastro e dei righelli. Importante fonte di errore è anche il fatto che le fotografie hanno nitidezza limitata e che la traccia lasciata dagli elettroni ha uno spessore.\\
Un'importatante causa di errore sistematico è sicuramente la presenza dell'apparato focalizzatore "Focus" che altera le condizioni di lavoro.\\
La misura di $\frac{e}{m} = 2.8 \pm 0.1 \pm 10\% \frac{10^{11} C}{Kg}$ non è in accordo con quella attesa. Si ritiene che ciò sia dovuto ad errori sistematici legati principalmente alle misure del potenziale acceleratore, che sembra in qualche modo correlato dalle altre funzioni dello strumento di misura, come ad esempio il potenziale $V_heat$.\\
E' stata comunque fornita una valida stima dell'ordine di grandezza di e/m.