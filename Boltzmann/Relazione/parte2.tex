\section{Convertitore RMS}
In \fig{Convertitore} è mostrato il terzo componente del circuito, che include condensatori di capacità C$_1 = 107 \pm 4$ nF, C$_2 = 9.2 \pm 0.4$ $\mu$F, C$_3 = 490 \pm 20$ nF. Abbiamo misurato ancora una volta l'ingresso e l'uscita, quest'ultima non più oscillante ma continua, per una frequenza fissata pari a $f = 1.85 \pm 0.03$ kHz, ottenendo V$_{in} = 2.10 \pm 0.07$ V (ampiezza picco-picco) e V$_{out} = 740 \pm 20$ mV, compatibile con il valore atteso di V$_{out,att} =\frac{V_{in}}{2\sqrt{2}} = 742 \pm 14$ mV. Aumentando la frequenza V$_{out}$ tende ad aumentare, rimanendo tuttavia entro l'incertezza, fino a circa 36 kHz, dopodichè cala. A frequenze basse, parimenti, V$_{out}$ cala. C'è quindi un range di frequenze per cui il convertitore funziona in modo ottimale.\\
Negli errori per le misure di tensione abbiamo considerato anche la parte di calibrazione, diversamente da quanto abbiamo fatto per i fit, dove l'errore, essendo di natura stocastica, non include la parte sistematica dovuta alla calibrazione.