\section{Scopo e strumentazione}

L'esperienza, divisa in due parti, è volta dapprima alla misura della lunghezza d'onda della riga gialla del sodio, a partire dall'angolo di diffrazione;la seconda prevede la stima della costante di Rydberg a partire dalle lunghezze d'onda delle righe dell'idrogeno.\\ La strumentazione include:
\begin{itemize} 
\item Uno spettroscopio a prisma (utilizzato nella prima parte), la cui struttura è mostrata in \fig{spettroscopio_prisma},dotato di un goniometro con risoluzione di 1' di grado.
\item Uno spettroscopio a reticolo di diffrazione (\fig{spettroscopio_reticolo}, utilizzato nella seconda parte), il cui goniometro ha risoluzione di 0.5' di grado.
\item Una lampada al cadmio, una lampada al mercurio, una all'idrogeno ed una al sodio.
\end{itemize}



\section{Misura della lunghezza d'onda della riga gialla del sodio}

\subsection{Calibrazione dello strumento con la lampada al cadmio}

Abbiamo posto la lampada al cadmio nella posizione indicata in \fig{spettroscopio_prisma}, in modo tale che la luce investisse completamente la fenditura.Lo spessore della fenditura è stato scelto in modo tale da non essere eccessivo e quindi compromettere la qualità della misura, ed al contempo troppo ridotto da impedire la visualizzazione di alcune bande. \newline
 Si è rimosso quindi il prisma e si sono allineati i due telescopi, ricercando la fenditura e centrandola con il telescopio mobile. Leggendo il goniometro abbiamo così potuto determinare un angolo di riferimento. \newline
  La misura dell'angolo in generale è affetta da tre tipi di errore: il primo è casuale, dovuto al posizionamento del telescopio, il secondo strumentale dovuto alla  risoluzione del goniometro, il terzo dovuto alla larghezza della banda di luce visualizzata. \\
Per stimare il primo tipo di errore ciascuna misura di angolo è stata effettuata indipendentemente dai componenti del gruppo e ne è stata trovata la semidispersione. Per il secondo tipo di incertezza si è considerata la risoluzione di un primo di grado. Per il terzo si è valutata metà larghezza della banda. \newline
 E' stata effettuata la misura dell'angolo di riferimento è si ottenuto: $\theta_{0}=203.067 \pm 0.03$ gradi. L'errore attribuito in questo caso è dovuto solamente alla larghezza della banda. Tuttavia nel seguito gli errori sono stati attribuiti in maniera differente. Ognuno dei componenti del gruppo ha individuato il centro della banda (che ha uno spessore di circa 2') ed ha effettuato la propria misura. Si è  considerata come misura dell'angolo la media delle tre misure; si è inoltre ritenuto opportuno attribuire come errore della misura solo la semidispersione delle tre misure e non la risoluzione dello strumento e la larghezza della banda(che non danno informazioni sul centro della banda)  \newline
 
Ricollocato il prisma nella sua posizione originaria, abbiamo dunque ruotato lo stesso in modo da formare un angolo di almeno 60 gradi con la normale. Ruotando il telescopio e il prisma abbiamo raggiunto la posizione di minima deviazione per la riga verde. La posizione del prisma non è più stata variata in seguito. Abbiamo così misurato gli angoli per le righe di emissione blu, azzurra, verde e rossa (\tab{calibrazione}). Si è svolto così un fit lineare al modello $y=mx+q$ a due parametri, che dà i seguenti risultati riportati anche in \fig{calibrazione_cadmio}:

\begin{itemize}
\item	$m_{fit}=-3195 \pm$ 40 nm$^{-1}$
\item	$q_{fit}=160.83 \pm 0.07$ gradi
\item	covarianza normalizzata = -0.99
\item	$\chi^{2}/ndof=3.15/2$
\item	probabilità associata $= 0.21$
\end{itemize}

\begin{table}[h]
	\centering
	\begin{tabular}{ *{3}{S[table-figures-exponent = 2]} }
		{$\lambda$ [\SI{}{1\per\nano\meter}]} & {$\theta$ [\SI{}{gradi}]} &{$\Delta\theta [\SI{}{gradi}]$} \\
		\midrule
		508.6 & 154.62 & 0.07 \\ 
		643.8 & 155.87 & 0.02 \\ 
		467.8 & 153.99 & 0.02 \\ 
		480.0 & 154.22 & 0.03 \\
	\end{tabular}
	\caption{lunghezza d'onda nominale e rispettivo angolo di rifrazione misurato.}
	\label{t:calibrazione}
\end{table}

Dal valore del $\chi^{2}$ possiamo affermare che l'andamento lineare è compatibile con i nostri dati.


Nel fit non è stato utilizzato l'angolo $\theta_{0}$. In questo modo gli angoli non sono quelli rispetto alla posizione di allineamento, tuttavia la calibrazione può essere effettuata comunque perché l'eventuale offset viene inglobato nella stima del termine noto fatta col fit.
La conoscenza dei valori nominali delle lunghezze d'onda ha un errore associato sufficientemente piccolo da poter essere trascurato nel fit.


\subsection{Misura con la lampada al sodio}

Sostituita la lampada al cadmio con la lampada al sodio, abbiamo misurato l'angolo a cui si è osservata la banda gialla, pari a $\theta_{s} = 155.49 \pm 0.03 $ gradi.L'errore è stato valutato attraverso la semidispersione su 5 misure effettuate dai componenti del gruppo. Utilizzando la dipendenza lineare con i parametri determinati nella sezione precedente, abbiamo così potuto calcolare il valore della lunghezza d'onda della riga gialla del sodio $\lambda = 598 \pm 3$ nm. L'errore è stato attribuito a partire da quelli sui parametri di fit e dalla loro covarianza.
