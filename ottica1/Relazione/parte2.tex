\section{Misura della costante di Rydberg}

\subsection{Misura del passo reticolare}
Si è utilizzato lo strumento nella configurazione in \fig{spettroscopio_reticolo}, con la lampada al mercurio in ingresso. E' stato possibile calcolare il passo reticolare dell'elemento diffrattore grazie alla relazione:
\begin{equation}
d = \frac{m\lambda}{sen\theta_{i} - sen\theta_{d}}
\label{legge_diffrazione}
\end{equation} dove $m$ è l'ordine di diffrazione, $\theta_{i}$ l'angolo di incidenza e $\theta_{d}$ l'angolo di diffrazione (tutti rispetto alla normale del reticolo). Nella misura è stata utilizzata la lunghezza d'onda nominale di ?? corrispondente al colore verde. E' stato scelto un angolo di incidenza ampio in modo tale da ridurre l'incertezza sui seni in ??formula 1 ??\newline
Dapprima è stato misurato l'angolo di riferimento corrispondente ai due telescopi allineati sull'asse ottico. In seguito sono stati calcolati gli angoli $\alpha_{0}$ e $\alpha_{d}$ di deflessione rispetto all'asse rispettivamente del fascio riflesso e il fascio relativo al primo ordine di diffrazione. Sono stati dunque ricavati i due angoli in ???relazione 1??, tramite le relazioni:
\begin{equation}
\theta_{i}=\frac{\pi-\alpha_{0}}{2}
\end{equation}
\begin{equation}
 \theta_{d}=\pi-\theta_{i}-\alpha{d}
\end{equation}
I valori ottenuti per gli angoli sono riportati in \tab{passo_reticolare}.
???tabella
Come nelle sezioni precedenti è stato determinato l'errore effettuando separatamente più misure di ciascun angolo e determinandone media e dispersione.
\newline
Dalla ??cito la 1?? si ottiene il valore  d=??. L'errore sul passo reticolare discende da quello sugli angoli trascurando invece quello sul'ultima cifra della lunghezza d'onda che risulta trascurabile.

\subsection{Righe spettrali dell'idrogeno}

Sostituendo alla lampada a mercurio la lampada a idrogeno, abbiamo misurato l'angolo di rifllessione e gli angoli di rifrazione per le bande viola e azzurra (primi due ordini) e per le bande verde e rossa (solo primo ordine). Qualora vi fosse più di una banda dello stesso colore, abbiamo misurato l'angolo per la più evidente. Noto $d$ e misurati gli angoli, abbiamo quindi calcolato $\lambda$ invertendo la ??formula 1. I risultati sono riportati in cito tabella 2???.\\

E' nota l'equazione di Rydberg:
\begin{equation}
\frac{1}{\lambda} = R (\frac{1}{n_{1}^2} - \frac{1}{n_{2}^2})
\end{equation} dove $n_1$ e $n_2$  sono due numeri quantici associati all'idrogeno ed R è la costante di Rydberg. Conoscendo preliminarmente l'ordine di grandezza della costante R si sono stimati i valori dei due numeri quantici per ciascuna lunghezza d'onda. E' stato eseguito un fit lineare numerico a due parametri y=mx+q, ottenendo: m=??, q=??; figura???.
L'intercetta risulta compatibile con il valore nullo e fornisce il valore m per la costante di Rydberg

\begin{table}[h]
\begin{center}
\begin{tabular}{| r @{$\pm$} l  @{ Rad } | r @{$\pm$} l  @{\quad nm } |}

\hline
\multicolumn{2}{c|}{$\alpha_{i}$} & \multicolumn{2}{c|}{$\lambda$} \\
\hline

\hline
\end{tabular}
\caption{spettro lampada a idrogeno.}
\end{center}
\end{table}
% sodio
