\section{Amplificatore differenziale e mediatore}
Abbiamo realizzato il circuito mostrato in \fig{Mediatore}, usando come componenti:
\begin{itemize}
\item $R_{18}=3.20 \pm 0.05[M\Omega]$
\item $R_{20}=1.17 \pm 0.01[M\Omega]$
\item $R_{21}=1.51 \pm 0.01[k\Omega]$
\item $R_{22}=38.6 \pm 0.4[k\Omega]$
\item $R_{23}=32.2 \pm 0.4[k\Omega]$
\item $R_{24}=38.7 \pm 0.4[k\Omega]$
\item $R_{26}=32.3 \pm 0.4[k\Omega]$
\item $C_5=232 \pm 12[nF]$
\item $C_6=50 \pm 2[nF]$
\end{itemize}
Ponendo il deviatore in posizione 0, abbiamo osservato all'oscilloscopio il segnale in posizione S9. Esso risultava sempre negativo, con un'ampiezza V$ = 1.1 \pm 0.1$ V. Dopo il mediatore (nella posizione Voltmetro) abbiamo ottenuto un segnale continuo, positivo, di ampiezza V$ = 1.12 \pm 0.05$ V.\\
Con il deviatore nella posizione 90$^\circ$, invece, abbiamo ottenuto in S9 V$_{S9} = 1.9 \pm 0.1$ V e dopo il mediatore V$_{out} = 224 \pm 10$ mV. Il segnale risultava circa a media nulla.\\
Considerato il circuito nel suo complesso, abbiamo ricercato, mantenendo il deviatore nella posizione 90$^\circ$, la posizione del potenziometro P2 che minimizzasse il segnale in uscita (in modulo), corrispondente ad un estremo (la stessa posizione che P2 aveva in precedenza). Variando il potenziometro, le parti positiva e negativa della forma d'onda aumentavano o diminuivano, in modo complementare, la loro estensione in un periodo.\\ Spostando poi il deviatore nella posizione 0$^\circ$, abbiamo misurato con il multimetro digitale la tensione in uscita, che appariva, come atteso, costante all'oscilloscopio, in funzione del numero di lastrine collocate tra il LED ed il fotodiodo (\tab{Fotodiodo}).\\
\begin{table}[h]
	\centering
	\begin{tabular}{cc}		
		{numero lastrine} & {V$_{out}$ [mV]}\\
                      0 & 1023\\
                      1 & 857\\
                      2 & 708\\
                      3 & 615\\
                      4 & 510\\
                      5 & 426\\
                      6 & 332\\
                      7 & 294\\
                      8 & 252\\
                      9 & 221\\
                     10 & 195\\
                     11 & 172\\
                     12 & 154\\
	           \midrule
	
 	\end{tabular}
	\caption{Tensione in uscita in funzione del numero di piastrine. }
	\label{t:Fotodiodo}
\end{table}
Abbiamo realizzato un fit con un'esponenziale a 2 parametri (la tensione iniziale ed il coefficiente di assorbimento del mylar, a moltiplicare l'opposto del numero di lastrine ad esponente), numerico. Come errori abbiamo considerato, per ciascuna misura di tensione, gli errori di lettura e calibrazione del multimetro sommati in quadratura. In \fig{Grafico} è mostrato il grafico con la funzione di fit.\\
Abbiamo ottenuto V$_{out,0} = 988 \pm 25$ mV, il coefficiente di assorbimento $b = 0.166 \pm 0.04$, una covarianza normalizzata $cov = 0.82$ e $\chi^2 = 842$ con 11 gradi di libertà. Il valore molto grande del $\chi^2$ può essere spiegato con la presenza di un errore ben più significativo di quelli strumentali dovuto alla variabilità della radiazione luminosa rilevata dal fotodiodo, che pur venendo coperto durante le misure, era raggiunto da un fondo di luce ambiente.

\section{Conclusioni}

