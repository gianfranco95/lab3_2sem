\section{Convertitore RMS}
In \fig{Convertitore} è mostrato il terzo componente del circuito, che include condensatori di capacità C$_1 = 107 \pm 4$ nF, C$_2 = 9.2 \pm 0.4$ $\mu$F, C$_3 = 490 \pm 20$ nF. Abbiamo misurato ancora una volta l'ingresso e l'uscita, quest'ultima non più oscillante ma continua, per una frequenza fissata pari a $f = 1.85 \pm 0.03$ kHz, ottenendo V$_{in} = 2.10 \pm 0.07$ V (ampiezza picco-picco) e V$_{out} = 740 \pm 20$ mV, compatibile con il valore atteso di V$_{out,att} =\frac{V_{in}}{2\sqrt{2}} = 742 \pm 14$ mV. Aumentando la frequenza V$_{out}$ tende ad aumentare, rimanendo tuttavia entro l'incertezza, fino a circa 36 kHz, dopodichè cala. A frequenze basse, parimenti, V$_{out}$ cala. C'è quindi un range di frequenze per cui il convertitore funziona in modo ottimale.\\
Negli errori per le misure di tensione abbiamo considerato anche la parte di calibrazione, diversamente da quanto abbiamo fatto per i fit, dove l'errore, essendo di natura stocastica, non include la parte sistematica dovuta alla calibrazione.

\section{Circuito completo}
Abbiamo collegato in serie i circuiti precedentemente realizzati, nell'ordine pre-amplificatore, passa-banda, post-amplificatore, convertitore RMS, mandando ogni volta l'uscita del precedente all'ingresso del successivo. Il valore di V$_{rms}$ è stato poi misurato con il multimetro digitale, mentre l'ingresso del pre-amplificatore è stato collegato a terra tramite una resistenza di valore variabile. Abbiamo così misurato V$_{rms}$ in funzione della resistenza; i valori ottenuti sono mostrati in \tab{resistenze}. L'errore sulle misure di tensione è dovuto alla fluttuazione delle misure e non all'errore sistematico del multimetro, molto minore.\\
\begin{table}[h]
	\centering
	\begin{tabular}{cccccccc}		
		 {R [k$\Omega$]}& {dR [k$\Omega$]}& {V$_{out}$ [mV]} & {dV$_{out}$ [mV]}\\
		 \midrule
                       217 & 2 & 352 & 10\\
                       147 & 1 & 270 & 10\\
                       102.0 & 0.8 & 230 & 10\\ 
                       56.0 & 0.5 & 173 & 5\\
                       32.8 & 0.3 & 137 & 7\\
                       22.0 & 0.2 & 117 & 6\\
                       15.2 & 0.1 & 102 & 5\\
                       9.78 & 0.08 & 90 & 4\\ 
                       2.70 & 0.02 & 68 & 5\\
                       1.75 & 0.01 & 66 & 3\\
                       0.328 & 0.003 & 60 & 5\\
                       0.118 & 0.001 & 60 & 5\\
 	\end{tabular}
	\caption{V$_{rms}$ in uscita in funzione della resistenza in ingresso.}
	\label{t:resistenze}
\end{table}
In \fig{resistenze} è mostrato un grafico del potenziale in funzione della resistenza. Su di esso è stato eseguito un fit numerico con la funzione attesa espressa in \ref{e:tot}, mantenendo come parametri V$_{0n}$, R$_T$, R$_n$. Come errori sono stati considerati gli errori sulle y. I risultati ottenuti dal fit sono:
\begin{itemize}
\item V$_{0n} = 59.6 \pm 0.9$ mV.
\item R$_T = 8.0 \pm 0.4$ $k\Omega$
\item R$_n = 8000 \pm 1400$ ${k\Omega}^2$.
\item $cov1 = 0.88$, $cov2 = -0.22$, $cov3 = 0.58$ (sono rispettivamente le covarianze normalizzate tra V$_{0n}$ e R$_T$, V$_{0n}$ e R$_n$, R$_T$ e R$_n$).
\item $\chi^2/dof = 1.3/9$.
\end{itemize}
\begin{equation}
V_{rms} = V_{0n} \sqrt{1+\frac{R}{R_T} + \frac{R^2}{R_n}}
\label{e:tot}
\end{equation}
Nota, dal prodotto delle amplificazioni tra pre-amplificatore e l'insieme di passa-banda e post-amplificatore, l'amplificazione totale A$_{tot} = 131000 \pm 16000$, nota la larghezza di banda $\Delta f$ precedentemente misurata e nota la temperatura T$ = 302 \pm 3$ K misurata con il tester, a partire da \ref{e:B} è possibile determinare la costante di Boltzmann k$_B = 2.2 \pm 0.6$ e-23 J/K, in accordo entro 2$\sigma$ con il valore atteso k$_{B,att} = 1.3806488 \pm 0.0000013$ e-23 J/K.
\begin{equation}
k_B = \frac{V_{0n}^2}{4 T {A_{tot}^2} \Delta f R_T}
\label{e:B}
\end{equation}

\subsection{Risposta in frequenza del circuito complessivo}
Sostituita la resistenza in ingresso, al fine di ridurre l'ampiezza del segnale in ingresso, ovvero di evitare saturazione in uscita, con un partitore di tensione avente una resistenza a massa R$_2 = 994 \pm 8$ k$\Omega$ e una all'ingresso R$_1 = 987 \pm 8$ k$\Omega$, abbiamo misurato la risposta in frequenza del circuito completo. In \tab{frequenza} sono mostrati i potenziali in uscita in funzione della frequenza, misurati all'oscilloscopio, ad un'ampiezza di ingresso costante V$_{in} = 19 \pm 5$ mV.\\
\begin{table}[h]
	\centering
	\begin{tabular}{cccccccc}		
		 {$f$ [kHz]}& {$df$ [kHz]}& {V$_{out}$ [V]} & {dV$_{out}$ [V]}\\
		 \midrule
                      4.5 & 0.1 & 0.392 & 0.008\\
                      5.2 & 0.1 & 0.584 & 0.008\\
                      5.5 & 0.1 & 0.80 &	0.02\\
                      5.6 & 0.1 & 0.94 & 0.04\\
                      5.9 & 0.1 & 1.20 & 0.02\\
                      6.0 & 0.1 & 1.38 & 0.02\\
                      6.1 & 0.2 & 1.56 & 0.02\\
                      6.2 & 0.2 & 1.60 & 0.02\\
                      6.3 & 0.2 & 1.70 & 0.02\\
                      6.7 & 0.2 & 1.64 & 0.04\\
                      6.8 & 0.2 & 1.30 & 0.02\\	
                      7.0 & 0.2 & 1.12 & 0.02\\
                      7.1 & 0.7 & 0.88 & 0.02\\
                      7.7 & 0.2 & 0.624 & 0.008\\
                      8.1 & 0.3 & 0.472 & 0.008\\
                      9.0 & 0.3 & 0.328 & 0.004\\
                      10.4 & 0.4 & 0.220 & 0.004\\
 	\end{tabular}
	\caption{V$_{rms}$ in uscita in funzione della frequenza in ingresso.}
	\label{t:frequenza}
\end{table}
In \fig{frequenza} è mostrato il plot di Bode. Un fit, eseguito con la funzione di trasferimento mostrata in \ref{e:passabanda}, a 3 parametri $\omega_i$, $\omega_r^2$, A dà come risultati:
\begin{itemize}
\item $f_0=\frac{\omega_i}{2\pi}= 6.33 \pm 0.02$ [kHz], in accordo con il valore ottenuto dalla risposta in frequenza del passa-banda $f_0 = 6.392 \pm 0.053$ [kHz].
\item A= $528 \pm 12$[MHz]
\item $\Delta \omega =-2\omega_r=  5.7 \pm 0.3$ [kHz], in accordo con $\Delta \omega = 6.1 \pm 0.6$ [kHz] ottenuto per l'insieme passa-banda e post-amplificatore.
\item $cov1 = -0.30$, $cov2 = 0.76$, $cov3 = -0.12$ (covarianze rispettivamente tra $\omega_r^2$ e $\omega_i$, $\omega_r^2$ e A, A e $\omega_i$).
\item $\chi^2/dof = 10.7/14$.
\end{itemize}
L'amplificazione in corrispondenza della frequenza di taglio è A$_0 = 92700 \pm 6300$, in accordo entro 3$\sigma$ con il valore dato dal prodotto di amplificazione del pre-amplificatore e amplificazione per il post-amplificatore A$_0 = 131000 \pm 16000$ e non in accordo con il valore atteso A$_0 = 245000 \pm 6000$.

\subsection{Risposta in ampiezza del circuito complessivo}
Abbiamo inoltre ricercato l'andamento di V$_{out}$(V$_{in}$) eseguendo le misure mostrate in \tab{risposta_ampiezza}. Un grafico è mostrato in \fig{risposta_ampiezza}. Un fit lineare presenta come risultati un'amplificazione $m = 10680 \pm 140$, non in accordo con il valore atteso (noto a partire dalla funzione di fit, posta alla frequenza di lavoro $f_L = 3.23 \pm 0.08$ kHz) $m_{att} = 9000 \pm 600$; un'intercetta $q = 10 \pm 9$ mV, $cov = -0.67$; $\chi^2 /dof = 0.2/3$.
\begin{table}[h]
	\centering
	\begin{tabular}{cccc}		
		 {V$_{in}$ [mV]} & {$dV_{in}$ [mV]} & {V$_{out}$ [mV]} & {$dV_{out}$ [mV]} \\
		 \midrule
                       11.8 & 4 & 132 & 12 \\
                       21.6 & 5 & 236 & 16 \\
                       40.8 & 6 & 464 & 20 \\
                       82 & 8 & 920 &40 \\
                       164 & 6 & 1760 & 40 \\
 	\end{tabular}
	\caption{Tensioni di uscita per vari valori della tensione di ingresso}
	\label{t:risposta_ampiezza}
\end{table}

\section{Conclusioni}
La presenza di numerose fonti di rumore non considerate inficia la qualità dell'esperienza; tuttavia, è evidente una leggera sovrastima della costante di Boltzmann. La sovrastima diviene significativa considerando amplificazione e larghezza di banda relative al circuito considerato nella sua interezza, poichè si ottiene  k$_B = 4.7 \pm 0.7$ e-23 J/K, che è ampiamente in disaccordo con il valore atteso. Considerando l'amplificazione attesa A$_0 = 245000 \pm 6000$ si ha però k$_B = 0.67 \pm 0.05$ e-23 J/K, molto inferiore al valore atteso.\\
L'apparato sperimentale non consente dunque di andare oltre una semplice stima dell'ordine di grandezza della costante di Boltzmann.