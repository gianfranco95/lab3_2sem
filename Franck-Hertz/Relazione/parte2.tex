\subsection{Generatore di rampa, acquisizioni misure}
Variando i valori minimi e massimi di $U_{a}$ si può osservare una variazione del numero di massimi e minimi della della curva della corrente di collettore in funzione del potenziale $U_{a}$. Il maggior numero di massimi visualizzabile è quello di 3, nel caso in cui $U_{A,min}= 0.0 \pm 0.5$ V, $U_{A,max}= 80.0 \pm 0.5$ V (in \fig{figura_3}) 
In queste condizioni sono esplorati tutti i valori di potenziali possibili e quindi, al variare del potenziale nel tempo di salita della rampa, tra le griglie sono presenti dapprima una sola banda luminosa, in seguito due ed infine tre.\\
Si è regolato il guadagno dell'amplificatore in modo tale che non fosse il più grande possibile, evitando che saturasse, in modo tale da poter visualizzare sull'oscilloscopio un segnale sufficientemente elevato.\\
Si è posto l'oscilloscopio in modalità due tracce(potenziale tra le griglie e potenziale proporzionale alla corrente di collettore in funzione del tempo). Sono state effettuate acquisizioni al variare del potenziale frenante.
Si è osservata la curva della corrente di collettore in funzione del tempo (e quindi del potenziale $U_{A}$), al variare del potenziale frenante.\\
 Il potenziale frenante $U_{E}$ è stato variato nel range $U_{E,min}=0.3\pm 0.1$V  $U_{E,max}=11.8 \pm 0.1$V . \\ 
Sono stati realizzati alcuni grafici che rappresentano i valori dei minimi, e dei massimi della curva della corrente, al variare del potenziale frenante: \fig{massimi_frenante}; \fig{minimi_frenante}.
Si osserva dai grafici l'andamento decrescente atteso. Inoltre si misura in corrispondenza dei minimi una corrente negativa per valori di potenziale frenante sufficientemente elevati. \\
E' stata effettuata inoltre una acquisizione tramite oscilloscopio della curva della corrente di collettore in funzione del potenziale tra le griglie, nelle stesse condizioni in cui era stata svolta in modalità manuale la misura iniziale. E' stato ottenuto: