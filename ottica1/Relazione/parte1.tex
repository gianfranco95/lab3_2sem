\section{Scopo e strumentazione}

L'esperienza, divisa in due parti, è volta dapprima alla misura della lunghezza d'onda della riga gialla del sodio, a partire dall'angolo di diffrazione, quindi viene stimata la costante di Rydberg a partire dalle lunghezze d'onda delle righe dell'idrogeno.\\ La strumentazione include uno spettroscopio a prisma (utilizzato nella prima parte), la cui struttura è mostrata in \fig{spettroscopio_prisma}, uno spettroscopio a reticolo di diffrazione (\fig{spettroscopio_reticolo}, utilizzato nella seconda), una lampada al cadmio, una lampada al mercurio, una lampada all'idrogeno ed una lampada al sodio. Il goniometro dello spettroscopio a prisma ha una risoluzione di 1' di grado; mentre per quello a reticolo la risoluzione è 0.5' di grado.

%\begin{figure}[h]
%\begin{center}
%\includegraphics[width=5cm]{}
%\caption{spettroscopio a prisma}
%\end{center}
%\end{figure}


\section{Misura della lunghezza d'onda della riga gialla del sodio}

\subsection{Calibrazione dello strumento con la lampada al cadmio}

Abbiamo posto la lampada al cadmio nella posizione indicata in \fig{spettroscopio_prisma}, in modo tale che la luce investisse completamente la fenditura. Si è tolto quindi il prisma, facendo attenzione a non toccarlo (avvalendoci dell'uso della vite) e abbiamo allineato i due telescopi, ricercando la fenditura e centrandola. Leggendo il goniometro abbiamo così potuto determinare un angolo di riferimento. La misura dell'angolo in generale è affetta da tre tipi di errore: il primo è casuale, dovuto al posizionamento del telescopio, il secondo strumentale dovuto alla  risoluzione del goniometro, il terzo dovuto alla larghezza della banda di luce visualizzata. \\
Per stimare il primo tipo di errore ciascuna misura di angolo è stata effettuata indipendentemente dai componenti del gruppo e ne è stata trovata la dispersione. Per il secondo tipo di incertezza si è considerata la risoluzione di un primo di grado. Per il terzo si è valutata metà larghezza della banda. Per l'angolo di riferimento si è ottenuto: $\theta_{0}=203.067 \pm 0.03$ gradi. L'errore è dovuto alla larghezza della banda ,tuttavia nel seguito abbiamo eseguito soltanto misure in cui si è cercato di individuare il centro della banda valutando statistica la dispersione dovuta all'errore umano.Perciò si è trascurato l'errore sulla larghezza della banda e anche quello strumentale.\\
 
Ricollocato il prisma nella sua posizione originaria, abbiamo dunque ruotato lo stesso in modo da formare un angolo di almeno 60 gradi con la normale. Ruotando il telescopio e il prisma abbiamo raggiunto la posizione di minima deviazione per la riga verde. La posizione del prisma non è più stata variata in seguito. Abbiamo così misurato gli angoli per le righe di emissione blu, azzurra, verde e rossa (\tab{calibrazione}), realizzando un grafico degli stessi in funzione dell'inverso della lunghezza d'onda, nominale, mostrato in \fig{calibrazione}. Si è svolto così un fit lineare al modello $y=mx+q$ a due parametri, che dà come risultati:	
\begin{itemize}
\item	$m_{fit}=	\SI{-3195 (40)}{\per\nano\meter}$
\item	$q_{fit}=	\SI{160.83 (0.07)}{\per\nano\meter}$
\item	$cov=-0.99$
\item	$\Chi^2/ndof=3.15/2$
\end{itemize}
\\
Dal valore del $\Chi^{2}$ possiamo affermare che l'andamento lineare è compatibile con i nostri dati.



	{}&{}
	 \midrule 
		$m_{fit}$	&	\si{-3195 \pm 40}{\per\nm}	\\
		$q_{fit}$	&	\si{160.83 \pm 0.07}{\per\nm} \\
		$cov$		&	-0.99	\\
		$\Chi^2/ndof $	&	$3.15/2$	\\	
	\end{tabular} 
	\caption{Misure relative all'impedenza in ingresso del circuito.} 
	\label{t:Z_in} 
\end{table}

Nel fit non è stato utilizzato l'angolo $\theta_{0}$. In questo modo gli angoli non sono quelli rispetto alla posizione di allineamento, tuttavia la calibrazione può essere effettuata comunque dato che siamo interessati alla pendenza della retta in \ref{calibrazione} e non al suo offset.



\subsection{Misura con la lampada al sodio}

Sostituita la lampada al cadmio con la lampada al sodio, abbiamo misurato l'angolo a cui si è osservata la banda gialla, pari a $\theta_{s} = ??$ rad. Utilizzando la dipendenza lineare con i parametri determinati nella sezione precedente, abbiamo così potuto calcolare il valore della lunghezza d'onda della riga gialla del sodio $\lambda = 598 \pm 3$ nm.
