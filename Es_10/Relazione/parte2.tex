\section{Circuito impulsatore}
Abbiamo utilizzato ArduinoNano per realizzare un circuito impulsatore, mostrato in \fig{Circuito}. Come componenti sono state utilizzate le resistenze $R_1 = 990 \pm 8$ $\Omega$, $R_2 = 990 \pm 8$ $\Omega$, $R_3 = 982 \pm 8$ $\Omega$, $R_4 = 977 \pm 8$ $\Omega$, un trimmer con resistenza massima $R_5 = 9.90 \pm 0.08$ k$\Omega$ ed i condensatori di capacità $C_1 = 106 \pm 4$ nF e $C_2 = 107 \pm 4$ nF.\\
Arduino permette di avere segnali digitali (onde quadre) alle uscite D12, D11, D10, D9, che funge da segnale di ingresso per l'integrato SN74LS244. Quest'ultimo restituisce in uscita i segnali Y1,Y2,Y3,Y4. I segnali di ingresso e di uscita sono stati visualizzati all'oscilloscopio.\\
In particolare abbiamo osservato che i segnali D11 e D12 mostravano uno sfasamento di 90 gradi, mentre i segnali D10 e D9, pur non sfasati rispetto a D12, presentavano una frequenza doppia. In uscita analogamente Y2 mostra uno sfasamento di 90 gradi rispetto ad Y1, mentre Y3 ed Y4 hanno una frequenza doppia ancora rispetto ad Y1. Come atteso, i massimi ed i minimi raggiunti dai potenziali in uscita erano rispettivamente minori e maggiori di quelli raggiunti dai potenziali in ingresso. Abbiamo misurato con i cursori i valori corrispondenti ai livelli logici 0 ed 1 per le uscite Y1 ed Y2, ottenendo Y1$_{min} = 138 \pm 6$ mV, Y1$_{max} = 3.1 \pm 0.1$ V, Y2$_{min} = 142 \pm 6$ mV, Y2$_{max} = 3.1 \pm 0.1$ V. Come errore abbiamo considerato la somma in quadratura di errore di lettura ed errore di calibrazione, come sempre per le misure di tensione all'oscilloscopio.\\
Agendo sul trimmer è possibile variare la frequenza del segnale in uscita; abbiamo quindi operato in modo da ottenere $f = 1.08 \pm 0.01$ kHz, calcolata a partire dalla misura del semiperiodo sull'oscilloscopio con errore di lettura, come sempre per le misure di tempi all'oscilloscopio, è dell'ordine del kHz come richiesto.
\subsection{Tempi di ritardo}
Abbiamo svolto la misura del tempo intercorso tra quando il segnale di ingresso raggiungeva la metà del suo valore massimo e quando il segnale di uscita faceva altrettanto, con ingresso D12 ed uscita Y1 (che sono in fase). Questo è $t_{PLH}$, nel caso in cui le forme d'onda sono in salita. Abbiamo inoltre misurato il tempo intercorso tra quando l'ingresso scendeva a metà del suo valore massimo e quando l'uscita faceva altrettanto, indicato come $t_{PHL}$ (con i fronti in discesa). Abbiamo ottenuto $t_{PLH} = 7 \pm 1$ ns e $t_[PHL] = 22 \pm 1$ ns. Ripetendo le misure anche per l'ingresso D11 e l'uscita Y2, pure in fase, abbiamo ottenuto $t_{PLH} = 14 \pm 1$ ns e $t_{PHL} = 22 \pm 1$ ns. Il datasheet prevedeva un valore massimo per entrambi i tempi di 18 ns; l'ordine di grandezza è comunque in accordo con quanto indicato.
\subsection{Tempi di salita e discesa}
Abbiamo infine misurato i tempi di salita $t_s$ e discesa $t_d$ (tempo impiegato dal segnale per salire dal 10 al 90 per cento del suo massimo o scendere dal 90 al 10) per i segnali Y1 e D12, ottenendo:
\begin{itemize}
\item $t_s = 11 \pm 1$ ns e $t_d = 8.8 \pm 0.4$ ns per D12
\item $t_s = 15.1 \pm 0.5$ ns e $t_d = 7.2 \pm 0.4$ ns per Y1
\end{itemize}
\section{Conclusioni}
Le osservazioni svolte sono in accordo con le aspettative.\\
Si è osservato che i circuiti a disposizione hanno una funzionalità che non solo rispetta i limiti indicati dal costruttore ma in diversi casi risulta anche migliore, al fine di realizzare circuiti logici. Fanno eccezione solamente due tempi di salita dll'integrato SN74LS244 che eccedono di qualche nanosecondo e le correnti di uscita della porta not che sono inferiori a quanto previsto.